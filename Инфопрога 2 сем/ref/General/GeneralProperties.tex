В AVL"=дереве высоты $h$ имеется не меньше $F_h$ узлов, где $F_h$ "--- число Фибоначчи.
Поскольку $F_n = \frac{(\frac{1 + \sqrt{5}}{2})^n - (\frac{1 - \sqrt{5}}{2})^n}{\sqrt{5}} =
\frac{\phi^n - (-\phi)^{-n}}{\phi - (-\phi)^{-1}}$,
где $\frac{\phi^n - (-\phi)^{-n}}{\phi - (-\phi)^{-1}}$ "--- золотое сечение,
то имеем оценку высоты AVL-дерева $h = Q(\lg(n))$,
где $n$ "--- число узлов. Следует помнить, что $Q(\lg(n))$ "--- мажоранта,
и её можно использовать только для оценки
(Например, если в дереве только два узла, значит в дереве два уровня,
хотя $\lg(2) = 1$. Для точной оценки глубины дерева следует использовать пользовательскую программу.

% листинг
\begin{}
function TreeDepth(Tree : TAVLTree) : byte;
begin
   if Tree <> nil then
      result := 1 + Max(TreeDepth(Tree^.left),TreeDepth(Tree^.right))
  else
      result := 0;
end;
\end{}

Тип Дерева можно описать так:

% ещё один листинг
\begin{}
TKey = LongInt;
TInfo = LongInt;
TBalance = -1..1;
TAVLTree = ^ TAVLNode;
  TAVLNode = record
    left, right : TAVLTree;
    key : TKey;
    info : TInfo;
{ Поле определяющее сбалансированность вершины }
    balance : TBalance;
  end;
\end{}