Относительно AVL"=дерева балансировкой вершины называется операция,
которая в случае разницы высот левого и правого поддеревьев $= 2$,
изменяет связи предок-потомок в поддереве данной вершины так,
что разница становится $ \leqslant 1$, иначе ничего не меняет.
Указанный результат получается вращениями поддерева данной вершины.

Используется 4 типа вращений:

\subsection*{Малое левое вращение}

%картинка 1

Данное вращение используется тогда,
когда (высота $b$"= поддерева; $L$ "--- высота )
$= 2$ и высота $С \leqslant$ высота $R$.

\subsection*{Большое левое вращение}

%картинка 2

Данное вращение используется тогда,
когда (высота $b$"=поддерева; $L$ "--- высота)
$= 2$ и высота $c$"=поддерева $>$ высота $R$.

\subsection*{Малое правое вращение}

%картинка 3

Данное вращение используется тогда,
когда (высота $b$"=поддерева "--- высота $R$)
$=$ 2 и высота $С <=$ высота $L$.

\subsection*{Большое правое вращение}

%картинка 4

Данное вращение используется тогда, когда (высота $b$"=поддерева; $R$ "--- высота)
$= 2$ ивысота $c$"=поддерева $ \greter $ высота $L$.
В каждом случае достаточно просто доказать то, 
что операция приводит к нужному результату и
что полная высота уменьшается не более чем на $1$ и не может увеличиться.
Из-за условия сбалансированности высота дерева $O(\lg(N))$,
где $N$ "--- количество вершин, поэтому добавление элемента требует $Q(\lg(N))$ операций.