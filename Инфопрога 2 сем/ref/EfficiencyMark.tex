Г.М.Адельсон-Вельский и Е.М.Ландис доказали теорему,
согласно которой высота AVL"=дерева с $N$ внутренними вершинами заключена между $log2(N+1)$ и $1.4404*log2(N+2)-0.328$,
то есть высота AVL"=дерева никогда не превысит высоту идеально сбалансированного дерева более,чем на $45\%$.
Для больших N имеет место оценка $1.04*log2(N)$.
Таким образом, выполнение основных операций $1"=3$ требует порядка $log2(N)$ сравнений.
Экспериментально выяснено, что одна балансировка приходится на каждые два включения и на каждые пять исключений.