Показатель сбалансированности в дальнейшем будем интерпретировать
как разность между высотой левого и правого поддерева,
а алгоритм будет основаться на типе TAVLTree, описанном выше.
Непосредственно при вставке (листу) присваивается нулевой баланс.
Процесс включения вершины состоит из трех частей:

\begin{enumerate}
    \item Прохода по пути поиска, пока не убедимся, что ключа в дереве нет.
    \item Включения новой вершины в дерево и определения результирующих показателей балансировки.
    \item "Отступления" назад по пути поиска и проверки в каждой вершине показателя сбалансированности.
    Если необходимо - балансировка
\end{enumerate}

Расширим список параметров обычной процедуры вставки параметром-переменной flag,
означающим, что высота дерева увеличилась.
Предположим, что процесс из левой ветви возвращается к родителю (рекурсия идет назад),
тогда возможны три случая:
{$h_1$ "--- высота левого поддерева, $h_r$- высота правого поддерева}
Включение вершины в левое поддерево приведет к

\begin{enumerate}
    \item $h_1 < h_r$: выровняется $h_1 = h_r$. Ничего делать не нужно.
    \item $h_1 = h_r$: теперь левое поддерево будет больше на единицу,
    но балансировка пока не требуется.
    \item $h_1 > h_r$: теперь $h_1 - h_r = 2$, "--- требуется балансировка.
\end{enumerate}

В третьей ситуации требуется определить балансировку левого поддерева.
Если левое поддерево этой вершины ($Tree^.left^.left$) выше правого ($Tree^.left^.right$),
то требуется большое правое вращение, иначе хватит малого правого.
Аналогичные (симметричные) рассуждения можно привести и для включение в правое поддерево.
Процедура вставки, предложенная Н.Виртом

%листинг
\begin{}
procedure TAVL.InsertNode(Var Tree : TAVLTree; const akey : TKey; const ainfo : TInfo; Var flag : Boolean);
 Var
 Node1, Node2 : TAVLTree;
 begin
 if Tree = nil then
  begin
    New(Tree);
    flag := true;
    with Tree^ do
      begin
        key := akey;
        info := ainfo;
        left := nil;
        right := nil;
        balance := 0;
      end;
    inc(AVL.FNodes);
  end
 else if Tree^.key > akey then
  begin
    InsertNode(Tree^.left,akey,ainfo,flag);
    if flag then
      case Tree^.balance of
      1 : begin Tree^.balance := 0; flag := false; end;
      0 : Tree^.balance := -1;
      -1 :   { Balance }
        begin
         Node1 := Tree^.left;
         if Node1^.balance = -1 then
          { LL }
            begin
              Tree^.left := Node1^.right;
              Node1^.right := Tree;
              Tree^.balance := 0;
              Tree := Node1;
            end
         else
            {LR}
            begin
              Node2 := Node1^.right;
              Node1^.right :=  Node2^.left;
              Node2^.left := Node1;
              Tree^.left := Node2^.right;
              Node2^.right := Tree;
              if Node2^.balance = -1 then Tree^.balance := 1 else Tree^.balance := 0;
              if Node2^.balance = 1 then Node1^.balance := -1 else Node1^.balance := 0;
              Tree := Node2;
            end;
         Tree^.balance := 0;
         flag := false
        end
      end
  end
 else if Tree^.key < akey then
  begin
    InsertNode(Tree^.right,akey,ainfo,flag);
    if flag then
      case Tree^.balance of
      -1 : begin Tree^.balance := 0; flag := false; end;
      0 : Tree^.balance := 1;
      1 :   { Balance }
        begin
         Node1 := Tree^.right;
         if Node1^.balance = 1 then
          { RR }
            begin
              Tree^.right := Node1^.left;
              Node1^.left := Tree;
              Tree^.balance := 0;
              Tree := Node1;
            end
         else
            {RL}
            begin
              Node2 := Node1^.left;
              Node1^.left :=  Node2^.right;
              Node2^.right := Node1;
              Tree^.right := Node2^.left;
              Node2^.left := Tree;
              if Node2^.balance = 1 then Tree^.balance := -1 else Tree^.balance := 0;
              if Node2^.balance = -1 then Node1^.balance := 1 else Node1^.balance := 0;
              Tree := Node2;
            end;
         Tree^.balance := 0;
         flag := false
        end
      end
  end
 end;
\end{}