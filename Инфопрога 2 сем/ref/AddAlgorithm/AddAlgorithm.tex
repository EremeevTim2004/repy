Показатель сбалансированности в дальнейшем будем интерпретировать
как разность между высотой левого и правого поддерева,
а алгоритм будет основаться на типе TAVLTree, описанном выше.
Непосредственно при вставке (листу) присваивается нулевой баланс.
Процесс включения вершины состоит из трех частей:

\begin{enumerate}
    \item Прохода по пути поиска, пока не убедимся, что ключа в дереве нет.
    \item Включения новой вершины в дерево и определения результирующих показателей балансировки.
    \item "Отступления" назад по пути поиска и проверки в каждой вершине показателя сбалансированности.
    Если необходимо - балансировка
\end{enumerate}

Расширим список параметров обычной процедуры вставки параметром-переменной flag,
означающим, что высота дерева увеличилась.
Предположим, что процесс из левой ветви возвращается к родителю (рекурсия идет назад),
тогда возможны три случая:
{$h_1$ "--- высота левого поддерева, $h_r$- высота правого поддерева}
Включение вершины в левое поддерево приведет к

\begin{enumerate}
    \item $ $: выровняется $ $. Ничего делать не нужно.
    \item $ $: теперь левое поддерево будет больше на единицу,
    но балансировка пока не требуется.
    \item $ $: теперь $ $ "--- $ $, - требуется балансировка.
\end{enumerate}

В третьей ситуации требуется определить балансировку левого поддерева.
Если левое поддерево этой вершины ($Tree^.left^.left$) выше правого ($Tree^.left^.right$),
то требуется большое правое вращение, иначе хватит малого правого.
Аналогичные (симметричные) рассуждения можно привести и для включение в правое поддерево.
Процедура вставки, предложенная Н.Виртом

%листинг
