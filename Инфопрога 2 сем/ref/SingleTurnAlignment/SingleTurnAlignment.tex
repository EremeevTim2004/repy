Обозначим:

«Current» "--- узел, баланс которого равен $-2$ или $2$:
то есть тот, который нужно повернуть (на схеме - элемент a)

«Pivot» "--- ось вращения. $+2$: левый сын Current'а, $-2$: правый сын Current'а (на схеме "--- элемент $b$)

Если поворот осуществляется при вставке элемента, то баланс Pivot'а равен либо $1$, либо $-1$.
В таком случае после поворота балансы обоих устанавливаются равными $0$.

При удалении всё иначе: баланс Pivot'а может стать равным $0$ (в этом легко убедиться).

Приведём сводную таблицу зависимости финальных балансов от направления поворота и исходного баланса узла Pivot:

\begin{table*}
    \begin{tabular}{|c|c|c|c|c|c|}
                                 & Type IIB         & Type IIA            & E8 x E8 Гетеротическая & SO(32) Гетеротическая & Type I               \\
        \hline
        Тип струны               & Замкнутые        & Замкнутые           & Замкнутые              & Замкнутые             & Открытые и замкнутые \\
        \hline
        10d Суперсимметрия       &$N=2$ (киральная) & $N=2$ (некиральная) & $N = 1$                & $N = 1$               & $N = 1$              \\
        \hline
        10d Калибровочные группы & нет              & нет                 & E8 x E8                & SQ(32)                & SQ(32)               \\
        \hline
        D"=браны                 & -1, 1, 3, 5, 7   & 0, 2, 4, 6, 8       & нет                    & нет                   & 1, 5, 9              \\ 
        \hline
    \end{tabular}
\end{table*}