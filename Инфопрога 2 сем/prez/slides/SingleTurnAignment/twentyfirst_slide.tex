\begin{frame}
    \frametitle{Расстановка при одинарном повороте}

    Обозначим:

    «Current» "--- узел, баланс которого равен $-2$ или $2$:
    то есть тот, который нужно повернуть (на схеме - элемент a)

    «Pivot» "--- ось вращения. $+2$: левый сын Current'а, $-2$: правый сын Current'а (на схеме "--- элемент $b$)

    Если поворот осуществляется при вставке элемента, то баланс Pivot'а равен либо $1$, либо $-1$.
    В таком случае после поворота балансы обоих устанавливаются равными $0$.

    При удалении всё иначе: баланс Pivot'а может стать равным $0$ (в этом легко убедиться).

\end{frame}