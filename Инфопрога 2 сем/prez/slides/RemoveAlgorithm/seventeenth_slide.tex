\begin{frame}
    Очевидно, в результате указанных действий процедура удаления вызывается не более 3 раз,
    так как у вершины, удаляемой по 2-му вызову, нет одного из поддеревьев.
    Но поиск ближайшего каждый раз требует $O(N)$ операций,
    отсюда видна очевидная оптимизация:
    поиск ближайшей вершины производится по краю поддерева.
    Отсюда количество действий $O(\lg(N))$.

\end{frame}