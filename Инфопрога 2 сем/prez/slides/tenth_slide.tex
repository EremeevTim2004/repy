\begin{frame}
    \frametitle{Большое правое вращение}

    %картинка 4
    \begin{figure}[ht]
        \includegraphics[width = \textwidth]{../images/4.gif}
        
        \caption{Схематическое изображение большого правого вращения}

    \end{figure}

    Данное вращение используется тогда, когда (высота $b$"=поддерева; $R$ "--- высота)
    $= 2$ ивысота $c$"=поддерева $ > $ высота $L$.
    В каждом случае достаточно просто доказать то, 
    что операция приводит к нужному результату и
    что полная высота уменьшается не более чем на $1$ и не может увеличиться.
    Из-за условия сбалансированности высота дерева $O(\lg(N))$,
    где $N$ "--- количество вершин, поэтому добавление элемента требует $Q(\lg(N))$ операций.
\end{frame}