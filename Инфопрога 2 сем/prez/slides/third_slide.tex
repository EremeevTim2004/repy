\begin{frame}
    \frametitle{Общие св-ва}
    
    В AVL"=дереве высоты $h$ имеется не меньше $F_h$ узлов, где $F_h$ "--- число Фибоначчи.
    Поскольку $F_n = \frac{(\frac{1 + \sqrt{5}}{2})^n - (\frac{1 - \sqrt{5}}{2})^n}{\sqrt{5}} =
    \frac{\phi^n - (-\phi)^{-n}}{\phi - (-\phi)^{-1}}$,
    где $\frac{\phi^n - (-\phi)^{-n}}{\phi - (-\phi)^{-1}}$ "--- золотое сечение,
    то имеем оценку высоты AVL-дерева $h = Q(\lg(n))$,
    где $n$ "--- число узлов. Следует помнить, что $Q(\lg(n))$ "--- мажоранта,
    и её можно использовать только для оценки
    (Например, если в дереве только два узла, значит в дереве два уровня,
    хотя $\lg(2) = 1$. Для точной оценки глубины дерева следует использовать пользовательскую программу.
\end{frame}