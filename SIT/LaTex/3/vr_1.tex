\section*{Вариант 1}
    
    % забыл выбрать зд, а там уже заполнено, не выкидывать же
    
    В левой полости давление постоянно $p_0$,
    а в правой полости оно имеет постояннкю составляющую и
    гармонически изменяющуюся по времени составляющую $p_0 + p_i^+(wt)$.
    Закон изменения давления на паравом торце  представим в виде

    \begin{equation}
        p_1^+ = p_{1m}^+ f_p(\omega t), f_p(\omega t) = \exp (i\omega t),
    \end{equation}

    где $p_{1m}^+$ "--- амплитуда пульсации давления на торце канала;
    $\omega$ "--- частота пульсации; $f_p(\omega t)$ "--- закон изменения давления.

    Введём в рассмотрение безразмерные переменные

    \begin{equation*}
        \psi = \delta_0 / l \ll 1,
        \lambda = w_m/\delta_0 \ll 1, 
        \Re  = \delta_0^2 \omega/v,
        \tau = \omega t, \xi = x/l,
        \zeta = z/\delta _0,
    \end{equation*}
    \begin{equation}
        V_z = w_m \omega U_\zeta,
        w = w_m W,
        u = u_m U,
        V_x = w_m\omega U_\xi /\psi,
    \end{equation}
    \begin{equation*}
        p = p_0 + w_m p v\omega(\delta_0\psi^2)^{-1}P,
        p^+ = w_m p v \omega (\delta_0 \psi^2)^{-1}P^+.
    \end{equation*}

    Здесь $p$ "--- давление; $\rho, v$ "--- плотность и нинематический коэффициент вязкости жидкости;
    $V_x, V_z$ "--- проекция скорости движения жидкости на оси координат, $w_m$ "--- амплитуда прогиба пластинны;
    $W$ "--- безразмерный прогиб пластины; $u_m$ "--- амплитуда продольного перемещения пластины;
    $U$ "--- бкзразмерное продольное перемещение пластины; $\psi, \lambda, \Re$ "--- параметры, характеризующие задачу. 