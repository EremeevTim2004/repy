Петров садится на коня и говорит, обращаясь к толпе, речь о том, что будет,
если на месте, где находится общественный сад, будет построен американский небоскреб.
Толпа слушает и, видимо, соглашается.Петров записывает что"=то у себя в записной книжечке.
Из толпы выделяется человек среднего роста и спрашивает Петрова, что он записал у  себя в записной книжечке.
Петров отвечает, что это касается только его самого.
Человек среднего роста наседает. Слово за слово, и начинается распря.
Толпа  принимает  сторону  человека среднего роста, и Петров, спасая свою жизнь, погоняет коня и скрывается за поворотом.
Толпа волнуется и, за неимением  другой жертвы, хватает человека среднего роста и отрывает ему голову.
Оторванная голова катится по мостовой и застревает в люке для водостока.
Толпа, удовлетворив свои страсти, расходится.