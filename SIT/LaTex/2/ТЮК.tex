Лето, письменный стол. Направо дверь. На столе картина. На картине нарисована лошадь, а в зубах у лошади цыган.
Ольга Петровна колет дрова. При каждом ударе с носа Ольги Петровны соскакивает  пенсне. Евдоким Осипович сидит в креслах и курит.
    О л ь г а  П е т р о в н а (ударяет колуном по полену, которое,  однако, нисколько не раскалывается).

    Е в д о к и м  О с и п о в и ч. Тюк!

    О л ь г а  П е т р о в н а. (Надевая пенсне, бьет по полену).
    
    Е в д о к и м  О с и п о в и ч. Тюк!
    
    Е в д о к и м  О с и п о в и ч (Надеваяпенсне). 
Евдоким Осипович!  Я вас прошу,  не говорите этого слова "тюк".
Евдоким Осипович. Хорошо, хорошо.

    О л ь г а  П е т р о в н а (Ударяет колуном по полену).

    Е в д о к и м  О с и п о в и ч. Тюк!

    О л ь г а  П е т р о в н а (надевая пенсне). 

Евдоким Осипович! Вы обещали не говорить этого слова "тюк".

    Е в д о к и м  О с и п о в и ч.  Хорошо,
хорошо, Ольга Петровна! Больше не буду.

    О л ь г а  П е т р о в н а (Ударяет колуном по полену).

    Е в д о к и м  О с и п о в и ч. Тюк!
    
    О л ь г а  П е т р о в н а (надевая пенсне)
Это безобразие! Взрослый пожилой человек
и не  понимает простой человеческой просьбы!

    Е в д о к и м   О с и п о в и ч. 
Ольга Петровна! Вы можете спокойно продолжать вашу
работу. Я больше мешать не буду.

    О л ь г а  П е т р о в н а . Ну  я прошу
вас, я очень прошу вас: дайте  мне расколоть
хотя бы это полено.

    Е в д о к и м  О с и п о в и ч.  Колите,
конечно, колите!

    О л ь г а  П е т р о в н а  (Ударяет колуном по полену).

    Е в д о к и м  О с и п о в и ч. Тюк!
Ольга  Петровна роняет колун,  открывает
рот, но ничего не может сказать. Евдоким
Осипович встает с кресел, оглядывает Ольгу
Петровну с головы до ног и медленно уходит.
Ольга Петровна стоит неподвижно с открытым
ртом и смотрит на удаляющегося Евдокима Осиповича.

\begin{center}
    Занавес медленно опускается.
\end{center}