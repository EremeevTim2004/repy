    Антон Михайлович плюнул, сказал <<эх>>, опять плюнул, опять сказал <<эх>>,
	опять плюнул, опять сказал <<эх>> и ушел. И Бог с ним. Расскажу лучше про
	Илью Павловича. Илья Павлович родился в 1883 году в Константинополе. Еще
	маленьким мальчиком его перевезли в Петербург, и тут он окончил немецкую
	школу на Кирочной улице. Потом он служил в каком"=то магазине, потом еще
	чего"=то делал, а в начале революции эмигрировал за границу. Ну и Бог с
	ним. Я лучше расскажу про Анну Игнатьевну. Но про Анну Игнатьевну
	рассказать не так-то просто. Во"=первых, я о ней почти ничего не знаю, а
	во"=вторых, я сейчас упал со стула и забыл, о чем собирался рассказывать.
	Я лучше расскажу о себе. Я высокого роста, неглупый, одеваюсь изящно и со
	вкусом, не пью, на скачки не хожу, но к дамам тянусь. И дамы не избегают
	меня. Даже любят, когда я с ними гуляю. Серафима Измайловна неоднократно
	приглашала меня к себе, и Зинаида Яковлевна тоже говорила, что она всегда
	рада меня видеть. Но вот с Мариной Петровной у меня вышел забавный случай,
	о котором я и хочу рассказать. Случай вполне обыкновенный, но все же
	забавный, ибо Марина Петровна благодаря мне совершенно облысела, как
	ладонь. Случилось это так: пришел я однажды к Марине Петровне, а она
	трах! "--- и облысела. Вот и все.

	\begin{flushright}
		9 - 10 июня 1941 года.
	\end{flushright}