Настоящее исследование изыскивали уточнить что глубокое учить и указать вне свои применения в настоящем мире.
Методы глубокого обучения продолжают прогрессировать, в частности, с использованием нескольких слоев.
Тем не менее, есть еще ограничения в использовании глубоких нейронных сетей, учитывая,
что они являются лишь одним из способов узнать несколько изменений, которые будут реализованы в входном векторе.
Изменения, предоставляемые целым рядом параметров, которые обновляются в период обучения.

Нельзя отрицать, что искусственный интеллект является более близкой реальностью, но ей не хватает долгий путь.
Принятие глубокого обучения в различных областях знаний позволяет обществу, в целом, извлекать пользу из чудес современных технологий.

Что касается искусственного интеллекта, то проверяется, что эта технология, способная к обучению, хотя очень важна линейная и нелинная природа человека,
которая представляет собой большой дифференциал и имеет важное значение для некоторых областей знаний, Это пока не может быть реализовано в глубоком обучении.

В любом случае, использование глубоких методов обучения позволит машинам для оказания помощи обществу в различных мероприятиях,
как показано, расширение когнитивных способностей человека и еще большее развитие в этих областях знаний.

