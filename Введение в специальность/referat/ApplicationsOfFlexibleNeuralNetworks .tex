Несмотря на существование нескольких классических техник, структуры глубокого обучения и ее основной единицы,
 нейрон является общим и очень гибким. Сделав сравнение с нейронами человека,
 который обеспечивает синапсы мы можем определить некоторые корреляции между ними. 

 % рисунок корреляция между нейрона человека и искусственной нейронной сетью 

 Отмечается, что нейрона формируется дендриты, которые являются точками входа, ядро, 
 которое представляет в искусственных нейронных сетей обработки ядра и точка выхода, что представлено аксон. 
 В обеих системах информация поступает, обрабатывается и изменяется. 

 Рассматривая его как математическое уравнение, нейрон отражает сумму входных данных, 
 умноженную на вес, и это значение проходит через функцию активации. 
 Эта сумма была выполнена Маккалок и Питтса в 1943.

 В связи с пресловутый интерес к глубокому обучению в настоящее время, Сантана считает,
 что это связано с двумя факторами, которые являются объем имеющейся информации и ограничение старых методов,
 кроме текущей вычислительной мощности для подготовки сетей Комплекс. 
 Гибкость соединения нескольких нейронов в более сложной сети является дифференциалом ГЛУБОКИХ учебных структур.
 Конволюционных нейронная сеть широко используется для распознавания лиц, обнаружения изображений и извлечения назначения.

 Обычная нейронная сеть состоит из нескольких слоёв, называемых слоями.
 В зависимости от вопроса, который должен быть решен количество слоев может варьироваться, 
 будучи в состоянии иметь до сотни слоев, являясь факторами,
 которые влияют на количество сложность проблемы, 
 времени и вычислительной мощности. 

 Есть несколько различных структур с бесчисленными целями,
 и их функционирование также зависит от структуры,
 и все они основаны на нейронных сетях.

 % рисунок примеры нейронных сетей

 Эта архитектурная гибкость позволяет глубже научиться решать различные вопросы.
 Глубокое обучение является общей объективной техникой, но самые передовые области были:
 компьютерное зрение, распознавание речи, обработка естественного языка, системы рекомендаций.

 Вычислительное зрение включает распознавание объектов, семантическую сегментирование,особенно автономные автомобили.
 Можно утверждать, что вычислительное зрение является частью искусственного интеллекта и определяется как набор знаний, 
 который направлен на искусственное моделирование человеческого видения с целью имитации его функций, 
 посредством разработки программного и аппаратного обеспечения Дополнительные.

 Среди применений вычислительного видения – военное использование, рынок сбыта, безопасность,общественные услуги и производственный процесс. 
 Автономные транспортные средства представляют собой будущее безопасного движения, но он все еще находится в стадии тестирования, 
 так как он включает в себя несколько технологий, применяемых к функции. Вычислительное зрение в этих автомобилях,
 так как позволяет распознавание пути и препятствий, улучшая маршруты.

 В контексте безопасности все чаще выделяются системы распознавания лиц,
 учитывая уровень безопасности в общественных и частных местах, 
 также внедрил в мобильных устройствах.
 Аналогичным образом они могут служить ключом к доступу к финансовым операциям,
 в то время как в социальных сетях, он обнаруживает присутствие пользователя или его друзей в фотографиях. 

 В отношении рынка сбыта, исследования, разработанного Image разведки указал,
 что 3 000 000 000 изображения распределяются ежедневно социальных сетей и 80\% 
 содержат указания, которые относятся к конкретным компаниям, но без текстовых ссылок.
 Специализированные маркетинговые компании предлагают услуги мониторинга и управления в режиме реального времени. 
 С технологией компьютерного зрения точность идентификации изображений достигает 99\%.

 В государственных услугах его использование покрывает безопасность сайта, отслеживая камеры, 
 трафик транспортного средства через стереоскопические изображения, 
 которые делают систему видения эффективной.

 В производственном процессе компании из разных отраслей используют 
 вычислительное зрение в качестве инструмента контроля качества.
 В любой отрасли, самое современное программное обеспечение,
 связанное с постоянно растущей вычислительной мощностью оборудования,
 увеличивает использование вычислительных возможностей. 

 Системы мониторинга позволяют признать заранее установленные стандарты и указывать на сбои, которые не могут быть идентифицированы при взгляде на работника на производственной линии.
 В этом же контексте, применительно к контролю складских запасов, используется проект автоматизации замещения.
 Инвентаризация и контроль продаж в режиме реального времени позволяют технологии контролировать деятельность данной компании, 
 увеличивая тем самым свою прибыль. Есть и другие приложения в области медицины, образования и электронной коммерции.