Глубокое обучение понимается как ветвь машинного обучения, основанная на группе алгоритмов,
которые стремятся формировать абстракций высокого уровня
данных с помощью глубокого графа с несколькими слоями обработки.
Состоит из нескольких линейных и нелинейных изменений.

Глубокое обучение работает компьютерная система для выполнения таких задач,
как распознавание речи, идентификация изображения и проекции.
Вместо того, чтобы организовывать информацию, чтобы действовать через заранее оговоренным уравнениям,
это обучение определяет основные шаблоны этой информации и
учит компьютеры развиваться путем идентификации моделей в обработке слоев.

Этот вид обучения является всеобъемлющей отраслью методов машинного обучения,
основанных на изучении представлений информации.
В этом смысле глубинное обучение представляет собой набор алгоритмов машинного обучения,
которые пытаются интегрировать несколько уровней,
которые являются признанными статистическими моделями, соответствующими разным уровням определений.
Более низкие уровни помогают определить многие понятия более высокого уровня.

Есть бесчисленное множество текущих исследований в этой области искусственного интеллекта.
Совершенствование методов глубокого обучения внедрило улучшения в способности компьютеров понять, что запрашивается.
Исследования в этой области направлена на поощрение более совершенных представлений и
разработать модели для выявления этих представлений из информации не помечены в больших масштабах,
некоторые в качестве основы в выводах неврологии и в интерпретации Обработка данных и
коммуникативные паттерны в нервной системе.
С 2006 года этот вид обучения возник как новый филиал исследований машинного обучения.

Недавно, новые методы были разработаны из глубокого обучения,
которые повлияли несколько исследований по обработке сигнала и идентификации шаблона.
Обратите внимание на ряд новых проблемных команд, которые могут быть решены с помощью этих методов,
в том числе машинного обучения и искусственного интеллекта ключевых точек.

Существует большое внимание средств массовой информации,
в соответствии с Yang, о достижениях, достигнутых в этой области.
Крупные технологические организации применили много инвестиций
в исследования глубокого обучения и их новых приложений.

Глубокое обучение включает в себя обучение на различных уровнях представительства и неосязаемости,
которые помогают в процессе понимания информации, образов, звуков и текстов.

Среди выставок, доступных на глубоком обучении, можно выделить два ярких момента. Первый показывает, что они являются моделями,
сформированные бесчисленными слоями или ступенями нелинейной обработки данных и также контролируются практикой обучения или нет,
о представлении атрибуций в более поздних и неосязаемых слоях.

Понятно, что глубокое обучение находится в суставах между ветвями исследований нейронной сети, AI,
графическое моделирование, идентификации и оптимизации шаблонов и обработки сигналов.
Внимание уделяется глубокому обучению в связи с улучшением мастерства обработки чипов,
значительным увеличением размера информации, используемой для обучения,
и последними достижениями в исследованиях в области машинного обучения и обработки сигналов.

Этот прогресс позволил практике глубокого обучения эффективно использовать сложные и нелинейные приложения,
определить представления распределенных и иерархических ресурсов,
и обеспечить эффективное использование Маркированные и немаркированные сведения.

Глубокое обучение относится к комплексному классу методов и проектов машинного обучения, которые объединяют характеристику 
использования многих слоев нелинейных обработанных данных иерархического характера. В связи с использованием этих методов и проектов,
большая часть исследований в этой области может быть классифицирована в трех основных наборов,в соответствии с Панг,
которые являются глубокими сетями для неконтролируемого обучения; Контролируется и гибрид.

Глубокие сети для неконтролируемого обучения доступны для задержания высокой последовательности корреляции анализируемой или идентифицируемой информации для проверки или ассоциации стандартов,
когда нет данных о стереотипах классов Доступно в базе данных.
Изучение атрибуции или неконтролируемого представительства относится к глубоким сетям. Кроме того,
вы можете искать назначение сгруппированных статистических дистрибутивов видимых данных и связанных с ним классов,
когда они доступны, и могут быть покрыты как часть видимых данных.

Глубокие нейронные сети для контролируемого обучения должны обеспечивать дискриминацию в отношении классификации,
обычно индивидуализации последующего распределения классов, связанных с видимой информацией,
которая всегда доступно для этого контролируемого обучения, также упоминается как глубокие дискриминационные сети.

Глубокие гибридные сети подсвечиваются дискриминацией, выявлемой с результатами генеративных или неконтролируемого глубоких сетей, которые могут быть достигнуты за счет
совершенствования и/или упорядочение сетей, находящихся под глубоким наблюдением. Его атрибуции также могут быть достигнуты,
когда дискриминационные руководящие принципы для контролируемого обучения используются для оценки стандартов в любой генеративной или неконтролируемой глубокой сети.

Глубокие и периодические сети являются моделями, которые представляют высокую производительность
в вопросах идентификации сомнительных моделей в Ivain и речи. Несмотря на свою силу представительства,
большие трудности в формировании глубоких нейронных сетей с общим использованием сохраняется и по сей день.
В связи с рецидивирующими нейронными сетями, исследования Хинтона инициировали формование слоями.

Настоящее исследование направлено на разъяснение прогресса глубокого обучения и его применения в соответствии с последними исследованиями.
Для этого будет проведено качественное описательное исследование с использованием книг, диссертаций, статей и веб-сайтов,
чтобы концептуально достижения в области искусственного интеллекта и особенно в глубоком обучении.

С прошлого десятилетия наблюдается растущий интерес к машинному обучению, учитывая, что существует постоянно увеличивающий взаимодействие между приложениями, будь то мобильные или компьютерные устройства, с физическими лицами, через программы для обнаружения спама, Распознавание фотографий в социальных сетях, смартфонов с распознаванием лица, среди прочих применений.
По данным Gartner все корпоративные программы будут иметь некоторые функции, связанные с машинного обучения до 2020 года.
Эти элементы направлены на обоснование разработки этого исследования.