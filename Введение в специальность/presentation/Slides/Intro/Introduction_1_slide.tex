\begin{frame}
    \frametitle{Введение}

    Глубокое обучение понимается как ветвь машинного, основанная на группе,
    которые стремятся формировать высокого уровня данных с помощью глубокого графа с несколькими слоями обработки. 
    Состоит из нескольких линейных и нелинейных изменений.

    Глубокое обучение работает компьютерная система для выполнения таких задач,
    как распознавание речи, идентификация изображения и проекции.
    Вместо того, чтобы организовывать информацию, чтобы действовать через заранее оговоренным уравнениям,
    это обучение определяет основные шаблоны этой информации и
    учит компьютеры развиваться путем идентификации моделей в обработке слоев.

    Этот вид обучения является всеобъемлющей отраслью методов машинного,
    основанных на изучении представлений информации.
    В этом смысле глубинное обучение представляет собой набор алгоритмов машинного,
    которые пытаются интегрировать несколько уровней,
    которые являются признанными статистическими моделями,
    соответствующими разным уровням определений.
    Более низкие уровни помогают определить многие понятия более высокого уровня.

    
\end{frame}