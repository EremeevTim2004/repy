\begin{frame}

    Настоящее исследование направлено на разъяснение прогресса глубокого обучения и его применения в соответствии с последними исследованиями.
    Для этого будет проведено качественное описательное исследование с использованием книг, диссертаций, статей и веб-сайтов,
    чтобы концептуально достижения в области искусственного интеллекта и особенно в глубоком обучении.

    С прошлого десятилетия наблюдается растущий интерес к машинному обучению, учитывая, что существует постоянно увеличивающий взаимодействие между приложениями, будь то мобильные или компьютерные устройства, с физическими лицами, через программы для обнаружения спама, Распознавание фотографий в социальных сетях, смартфонов с распознаванием лица, среди прочих применений.
    По данным Gartner все корпоративные программы будут иметь некоторые функции, связанные с машинного обучения до 2020 года.
    Эти элементы направлены на обоснование разработки этого исследования.

\end{frame}