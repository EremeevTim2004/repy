\begin{enumerate}
    \item Найдём долю кинетической энергии пули $\chi$,
    переходящей в кинетическую энергию системы маятник-пуля.
    По закону сохранения и превращения энергии при абсолютно неупругом ударе пули  маятник имеем

    \begin{equation} 
            \frac{mv^2}{2}=\frac{(M+m)V^2}{2}+Q 
    \end{equation},

    где $Q$ "--- часть кинетической энергии пули, перешедшая в теплоту.
    Тогда с учётом соотношения (1а) на основании (3) получаем

    \begin{equation} 
            \chi=\frac{\frac{(M+m)V^2}{2}}{\frac{mv^2}{2}}=\frac{m}{M+m}
    \end{equation}

    Поскольку $M \gg m$, то лишь очень незначительная часть кинетической энергии пули переходит в кинетическую энергию маятника с пулей.

    \item При абсолютно упругом ударе пули о маятник кинетическая энергия пули не переходит в теплоту,
    и кинетическая энергия системы маятник-пуля остаётся постоянной;
    количество движения, приобретаемое маятником при таком ударе, выражается

    \begin{equation} 
            MV^\prime=m(2v-V^\prime)
    \end{equation},

    откуда
                    
    \begin{equation} 
            v=\frac{1}{2}\frac{M+m}{m}V^\prime 
    \end{equation},

    где $V^\prime$ "--- начальная скорость маятника после абсолютно упругого удара с пулей.
    Следовательно, при абсолютно упругом ударе скорость маятника $V^\prime$ и баллистический отброс $S_0$ в два раза больше,
    чем при абсолютно неупругом ударе.

    В случае выражение для скорости пули примет вид

    \begin{equation} 
            v=\frac{M+m}{m}\frac{S_0}{2}\sqrt{\frac{g}{l}} 
    \end{equation}

    \item Соотношение (2) для начальной скорости маятника $V$ в момент удара можно получить, не используя закон сохранения механической энергии.

    Поскольку продолжительность удара очень мала, то после удара при малом угле отклонения маятника продолжит своё движение по гармоническому закону

    \begin{equation} 
            s=S_0\sin\frac{2\pi}{T}t
    \end{equation},

    где $S_0$ "--- амплитуда колебания,
                    
    \begin{equation} 
            T=2\pi\sqrt{\frac{g}{l}}
    \end{equation} "--- период колебания

    Мгновенное и максимальное значение скорости маятника соответственно запишутся следующими соотношениями:
                    
    \begin{equation} 
            v=\frac{dS}{dt}=\frac{2\pi}{T}S_0\cos\frac{2\pi}{T}t
    \end{equation}
                    

    \begin{equation} 
            V_0=\frac{\partial\pi}{T}S_0=S_0\sqrt{\frac{g}{l}}
    \end{equation}

    Очевидно, что при выполнении равенства (1) начальная скорость маятника будет практически равна максимальной скорости $V_0$,
    с которой маятник должен проходит положение равновесия при гармоническом движении,
    т.е. будет выполняется совпадающее с соотношением (2) приближенное равенство
                    
    \begin{equation} 
            \frac{\int_{0}^{\tau} F \,dt }{M+m}\simeq S_0\sqrt{\frac{g}{l}} 
    \end{equation}
                    
    или
                    
    \begin{equation} 
            \int_{0}^{\tau} F \,dt\simeq M^\prime S_0\sqrt{\frac{g}{l}}
    \end{equation},

    где $M^\prime=M+m$.

    Из равенства (4) видим, что импульс силы действует на маятник по линейному закону.
    Это свойство метода измерения очень ценно на практике.

    Выполнение соотношения, подобного выражению (4),
    при воздействии импульса силы на покоящийся маятник лежит в основе баллистического метода.
\end{enumerate}