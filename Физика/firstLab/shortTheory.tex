Измерение ряда физических величин может быть сведено к измерению пропорционального им импульса силы.
Если на находящийся в равновесии маятник взаимодействовать импульсом силы так,
что маятник за время действия силы не успевает существенно отклониться от положения равновесия,
то первое максимальное отклонение маятника от полученного толчка "--- баллистический отброс пропорционально импульсу силы.
Следовательно, измерение физической величины тогда можно свести к измерению баллистического отброса.
В этом и состоит баллистический метод измерения.

Метод баллистического маятника позволяет свести измерение скорости пули
к измерению отклонения сравнительно медленно движущегося маятника после
абсолютно неупругого удара с пулей.

Для упрощения расчётов баллистический маятник выполняют в таком виде,
чтобы его можно было рассматривать как математический.
Пуля массы $m$, летящая со скоростью $\vec{v}$, попадает в покоящийся маятник массы $M$.
Застревает в нём и сообщает общей массе $M + m$ некоторую начальную скорость $\vec{v}$,
в результате чего маятник с пулей отклоняется на расстояние $S_0$.

Для проведения измерения с баллистическим маятником необходимо чтобы закон сохранения количества движения мог быть выражен в следующем виде:

\begin{equation}
        \vec{v}=(M+m)\vec{V}
\end{equation},

т.е. необходимо, чтобы сразу после удара был в точности равен по модулю и по направлению вектору количества движения пули непосредственного удара.

Как известно, закон сохранения количества движения справедлив только для замкнутой системы тел, для которой сумма внешних сил равна нулю.
Для системы маятник-пуля внешние силы это сила тяжести, сила натяжения нитей, а так же мгновенная ударная сила, возникающая в общем случае в точке подвеса маятника во время удара.
Силой сопротивления воздуха пренебрегаем. Во время удара и после него система эта система становится незамкнутой, т.к. внешние силы, действующие на маятник с пулей, нескомпенсированы и сумма их не равна нулю.

Выполнение во время удара закона сохранения количества движения в виде равенства обеспечивает выполнение следующих условий:

\begin{enumerate}
        \item Вектор скорости пули в момент удара должен быть направлен по прямой,
        проходящей через центр тяжести маятника.
        При не выполнении этого условия часть импульса ударной силы
        $\int_{0}^{\tau } \vec{F} \,dt=m\vec{v}$ 
        будет передаваться точке подвеса маятника.

        \item Вектор должен быть направлен перпендикулярно плоскости,
        в которой лежат ось качания и точка центра тяжести покоящегося маятника,
        т.е. в направление оси $x$.
        В противном случае, маятнику будет сообщаться вращательное движение относительно других осей, помимо оси $AA'$.

        \item Положительность импульса
        $\int_{0}^{\tau } \vec{F} \,dt$
        должна быть настолько мало,
        чтобы маятник к концу удара не успевал существенно отклонится от положения равновесия.
\end{enumerate}

Практически 3-е условие обеспечивается выбором достаточно длинной нити подвеса и высокой вязкостью вещества в маятнике.

При выполнении перечисленных условий скорость пули находится из равенства (1).

\begin{equation}
        v=\frac{M+m}{m}V 
\end{equation}

Полагая после удара систему маятник-пуля-Земля консервативной,
то если пренебрегая рассеянием энергии,
примененим к ней закон сохранения механической энергии.

При максимальном отклонении маятника скорость $V$ обратится в нуль и кинетическая энергия маятника полностью перейдёт в его потенциальную, т.е.
        
\begin{equation}
        \frac{(M+m)V^\prime}{2}=(M+m)gh
\end{equation}

Тогда

\begin{equation} 
        V=\sqrt{2gh}
\end{equation},

где $ h $ "--- наибольшая высота подъёма центра тяжести маятника с пулей,

$ g $ "--- ускорение свободного подения.

Поскольку выполяется равенство

\begin{equation}
        h=l-l\cos\alpha =2\sin ^2\frac{\alpha }{2} 
\end{equation},

где $\alpha$ "--- максимальный угол отклонения,
$l$ "--- растояние от оси вращения до центра тяжести маятника,
и учитывая, что ввиду малости угла $\alpha$ можно предположить

\begin{equation} 
        \sin \alpha =\frac{S_0}{l}\simeq\alpha\simeq 2\sin\frac{\alpha}{2}  
\end{equation}

То для скорости маятника имеем выражение

\begin{equation} 
        V=2\sqrt{gl}\sin\frac{\alpha}{2}=S_0\sqrt{\frac{g}{l}} 
\end{equation}

Подставив значения скорости в формулу (1a), получим:

\begin{equation} 
        v=\frac{M+m}{m}S_0\sqrt{\frac{g}{l}}
\end{equation}