Итак, скорость каждой из пуль равна:

\begin{equation*} 
        \overline{v_\text{мал}} = 2,95 \pm 0,5 \text{ м/с; }
        \delta_{v_\text{мал}} = \pm \frac{0,5}{2,95} \cdot 100 \% = 16,95 \%
\end{equation*}

\begin{equation*} 
        \overline{v_\text{ср}} = 4,38 \pm 0,32 \text{ м/с; }
        \delta_{v_\text{ср}} = \pm\frac{0,32}{4,38}\cdot 100 \% = 7,3 \%
\end{equation*}

\begin{equation*} 
        \overline{v_\text{бол}} = 5,56 \pm 0,22 \text{ м/с; }
        \delta_{v_\text{бол}} =\pm\frac{0,22}{5,56}\cdot 100 \% = 3,96 \%
\end{equation*}

Найдём среднюю относительную погрешность измерений:

\begin{equation*}
				\overline{\delta v} = \frac{\delta_{v_\text{мал}} + \delta_{v_\text{ср}} + \delta_{v_\text{бол}}}{3} = 
				\frac{16,95 \% + 7,3 \% + 3,96 \%}{3} = 9,4 \%
\end{equation*}

Подсчитаем также максимальную относительную погрешность метода измерений, которая вычисляется следующим образом:

\begin{equation*}
        \frac{\Delta v}{v}=\frac{\Delta M}{M}+\frac{\Delta m}{m}+\frac{1}{2}\frac{\Delta l}{l}+\frac{1}{2}\frac{\Delta g}{g}+\frac{\Delta S_0}{S_0}
\end{equation*}

\begin{equation*} 
        \frac{\Delta v_\text{мал}}{v_\text{мал}} = \text{ 0,26; }
        \frac{\Delta v_\text{ср}}{v_\text{ср}} = \text{ 0,17; }
        \frac{\Delta v_\text{бол}}{v_\text{бол}} = \text{ 0,14}
\end{equation*} 

Найдём среднюю погрешность метода по всем пулям:

\begin{equation*}
        \overline{\frac{\Delta v}{v}} = \frac{0,26 + 0,17 + 0,14}{3} = 0,19 \text{ или 19\%}
\end{equation*}