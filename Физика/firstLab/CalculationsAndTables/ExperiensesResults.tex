Итак, скорость каждой из пуль равна:

\begin{equation*} 
        \vec{v_\text{мал}} = 5,3 \pm 1,01 \text{ м/с; }
        \delta_{v_\text{мал}} = \pm \frac{1,01}{5,3} \cdot 100 \approx 19 \%
\end{equation*}

\begin{equation*} 
        \vec{v_\text{ср}} = 4,76 \pm 0,5 \text{ м/с; }
        \delta_{v_\text{ср}} = \pm\frac{0,5}{4,76}\cdot 100 \approx 10,5 \%
\end{equation*}

\begin{equation*} 
        \vec{v_\text{бол}} = 4,63 \pm 0,146 \text{ м/с; }
        \delta_{v_\text{бол}} =\pm\frac{0,146}{4,63}\cdot 100 \approx 3 \%
\end{equation*}

\begin{equation*}
        \vec{\Delta v} =
        \frac{\frac{\Delta v_\text{мал}}{v_\text{мал}} + 
        \frac{\Delta v_\text{ср}}{v_\text{ср}} + 
        \frac{\Delta v_\text{бол}}{v_\text{бол}}}{3} = 
        \frac{19 + 10,5 + 3}{3} = 10,8 \%
\end{equation*}

Подсчитаем также максимальную относительную погрешность метода измерений, которая вычисляется следующим образом:

\begin{equation*}
        \frac{\Delta v}{v}=\frac{\Delta M}{M}+\frac{\Delta m}{m}+\frac{1}{2}\frac{\Delta l}{l}+\frac{1}{2}\frac{\Delta g}{g}+\frac{\Delta S_0}{S_0}
\end{equation*}

\begin{equation*} 
        \frac{\Delta v_\text{мал}}{v_\text{мал}} = \text{ 25,33; }
        \frac{\Delta v_\text{ср}}{v_\text{ср}} = \text{ 14,7; }
        \frac{\Delta v_\text{бол}}{v_\text{бол}} = \text{ 11,66;}
\end{equation*} 
