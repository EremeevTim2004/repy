\begin{enumerate}
    \item Измерить на весах массы пуль и съёмного внутреннего цилиндра маятника.

    \item Отрегулировать длину нитей так, чтобы геометрическая ось маятника была направлена горизонтально по направлению ствола пушки.

    \item Установить шкалу параллельно оси маятника вблизи визира маятника.
    
    \item Сжать пружину пушки и зафиксировать штифтом её положение. Вставить пулю в дуло пушки и дослать её шомполом до упора.
    
    \item Поднятием штифта произвести выстрел и снять отсчёт горизонтального смещения маятника по шкале.
    
    \item С каждой пулей произвести не менее пяти выстрелов. Опыт производить с тремя пулями различного веса.
    
    \item По рабочей формуле $  $ посчитать скорость пули при каждом выстреле. %хз что за формула
    Для каждой пули высчитать среднее значение скорости пули и среднюю абсолютную погрешность измерения.
    Данные наблюдения и расчётов занести в таблицу.
    
    \item Окончательный результат для каждой пули записать в виде:
    
    \begin{equation} 
            v = \overline{v} \pm \overline{\Delta v}
    \end{equation}
    \begin{equation} 
            \delta_v=\pm\frac{\overline{\Delta v}}{\overline{v}}\cdot 100 \%
    \end{equation}

    \item Подсчитать максимальную относительную погрешность метода измерений по формуле
    $\frac{\Delta v}{v}=\frac{\Delta M}{M}+\frac{\Delta m}{m}+\frac{1}{2}\frac{\Delta l}{l}+\frac{1}{2}\frac{\Delta g}{g}+\frac{\Delta S_0}{S_0}$,
    где в качестве погрешностей измерений следует подставлять погрешности отсчёта средств измерений.

    \item Сравнить полученное значение со значением относительной погрешности результата измерений.
\end{enumerate}