Наше современное представление о вселенной и ее происхождении зависит не только от фундаментальных законов физики, но и от начальных условий во времена большого взрыва.
Например, движение брошенного мяча определяется законами гравитации. Однако, имея лишь законы гравитации, нельзя предсказать,
где упадет мяч. Нужно еще знать начальные условия, то есть величину и направление его скорости в момент броска.
Для описания начальных условий, существовавших при рождении Вселенной, используется модель Большого взрыва.
В стандартной модели Большого взрыва начальные условия задаются бесконечными значениями энергии, плотности и температуры в момент рождения вселенной.
Иногда пытаются представить этот момент истории как взрыв некоей космической бомбы, порождающей материю в уже существующей Вселенной.
Однако этот образ несправедлив, так как когда взрывается бомба, она взрывается в определенном месте пространства и в определенный момент времени и ее содержимое просто разлетается в разные стороны.
Большой взрыв представляет собой порождение самого пространства. В момент Большого взрыва не было никакого пространства вне области взрыва.
Или, если быть более точным, еще не было нашего пространства, возникавшего как раз в процессе взрыва и инфляционного расширения

Теория струн модифицирует стандартную космологическую модель в трех ключевых пунктах.
Во-первых, из теории струн следует, что вселенная в момент рождения имела минимально допустимый размер.
Во-вторых, из теории струн следует дуальность малых и больших радиусов.
В-третьих, число пространственно"=временных измерений в теории струн и М"=теории больше четырех, поэтому струнная космология описывает эволюцию всех этих измерений.
В начальный момент существования Вселенной все ее пространственные измерения равноправны и свернуты в многомерный клубок планковского размера.
И только потом, в ходе инфляции и Большого взрыва часть измерений освобождается из оков суперструн и разворачивается в наше огромное 4"=мерное пространство"=время.

Из теории струн (дуальности больших и малых размеров) следует, что сокращение радиусов пространств до и ниже планковского размера физически эквивалентно уменьшению размеров пространства до планковских, с последующим их увеличением.
Поэтому сжатие Вселенной до размеров, меньших планковских, приведет к прекращению роста температуры и ее последующему снижению, как после Большого взрыва, с точки зрения внутреннего наблюдателя, находящегося в этой вселенной.
Получается достаточно интересная картина, чем"=то напоминающая пульсирующую вселенную, когда одна вселенная через своеобразный коллапс до клубка планковских размеров разворачивается затем в новую расширяющуюся Вселенную с теми же, по сути, физическими свойствами.

Теория суперструн активно развивается в последнее время, поскольку она может правильно описать всю нашу физику на всех энергетических масштабах.
В ней есть все "--- квантовая физика, фермионы и бозоны, калибровочные группы и гравитация.
В последние несколько лет произошел настоящий прорыв в понимании сути теории, включая D"=браны и дуальность.
Струнная теория успешно применяется к исследованию черных дыр и квантовой гравитации.
Хотя, как было упомянуто выше, до полного понимания теории еще далеко.