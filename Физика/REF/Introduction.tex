Первоначальной основой любой физической теории служат наблюдения,
и успех или неудача теории зависит от степени совпадения
теоретических выкладок с наблюдениями и экспериментами.
Однако по мере продвижения науки в область более фундаментальных явлений,
которые невозможно непосредственно наблюдать,
значительную роль начинает играть математическая структура теории.
Теория, обобщающая то, что известно о мире на сегодняшней день,
все равно была бы не совсем общей. Она бы лишь отыскивала наиболее фундаментальные объекты,
пытаясь с их помощью  объяснить единую природу четырех известных взаимодействий (сильного, слабого, электромагнитного и гравитационного)

Стандартная Модель описывает большинство явлений,
которые мы можем наблюдать с использованием современных технических средств,
но многие вопросы Природы остаются без ответа.
Цель современной теоретической физики состоит в объединении описаний всех процессов вселенной.
Исторически, этот путь довольно удачен. Например,
Специальная Теория Относительности Эйнштейна объединила электричество и магнетизм в электромагнитную силу.
В работе Глэшоу, Вайнберга и Салама, получившей Нобелевскую премию 1979 года, показано,
что электромагнитное и слабое взаимодействия могут быть объединены в электрослабое.
Сегодня есть все основания полагать, что все силы в рамках Стандартной Модели в конечном итоге объединяются.
Сравнивая сильное и электрослабое взаимодействия, нам придется уйти в область больших энергий,
и эти взаимодействия  сравняются по силе в районе $10^16$ ГэВ.
Гравитация также сравняется с ними при энергиях порядка $10^{19}$ ГэВ. 

Цель теории струн состоит в объяснении объединения взаимодействий.