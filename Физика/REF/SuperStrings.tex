Существует теория, базирующаяся на предыдущей и основанная на суперсимметрии.
Чтобы понять, в чем она заключается, нужно уяснить смысл термина «измерение».
Под измерением понимают некие характеристики системы.
Классический пример "--- кубики разных цветов.
Цвет можно принять за дополнительное измерение к общеизвестным трём "--- высоте, длине и ширине.
Симметрия "--- это инвариантность относительно некоторых преобразований.
С повышением температуры системы уровень её симметричности повышается.
Иначе говоря, растет хаотичность, неупорядоченность и уменьшается число параметров, пригодных для описания этой системы.
Таким образом, теряется информация, которая позволяет различить две любые точки внутри системы.
Например, на ранних этапах своей жизни физическая вселенная была очень горячей (ее температура была миллионы миллиардов градусов) и в ней существовала симметрия,
но с понижением температуры (сейчас средняя температура вселенной около трёх градусов по Кельвину) симметричность нарушается. 

Все «элементарные» частицы делятся на два класса — бозоны и фермионы.
Первые, например фотон и гравитон, могут собираться вместе в большие скопления,
в отличие от них каждый фермион должен подчиняться принципу Паули.
К фермионам относится в частности электрон.
Различия физического поведения разных типов частиц требуют различного математического описания.

И бозоны, и фермионы могут сосуществовать в одной физической системе,
и такая система может обладать особым видом симметрии "--- суперсимметрией.
Она отображает бозоны в фермионы и обратно.
Для этого, естественно, требуется равное количество обоих видов частиц,
но этим условия суперсимметрии не ограничиваются.
Суперсимметричные системы могут существовать только в так называемом суперпространстве.
Оно отличается от обычного пространства"=времени наличием называемых фермионных координат
и преобразования суперсимметрии в нем похожи на вращения и сдвиги в обычном пространстве.
В суперпространстве частицы и поля представляются набором частиц и полей обычного пространства,
со строго фиксированным количественным соотношением бозонов и фермионов и их характеристик (спин и т. п.).
Входящие в такой набор частицы"=поля называют суперпартнёрами.

Суперпартнеры «сглаживают» друг друга.
Это явление, наряду с особенностями геометрии суперпространств,
значительно затрудняет объяснение процессов,
происходящих в суперпространствах,
с точки зрения квантовой теории.
Струны, существующие в суперпространстве,
называются суперструнами.
Иными словами, струна в обычном пространстве,
на мировом листе которой существует определенный
набор фермионных полей, и есть суперструна. 

Суперсимметрия накладывает определенные ограничения на поведение суперструн.
В суперпространстве не может возникнуть тахионов,
так как из-за его свойств у тахиона не может быть суперпартнера.
Кроме того, благодаря суперсимметрии,  возникает такое состояние,
в котором суперструна избавлена от противоречий.
Размерность такого пространства оказывается равной 10.
Причем фермионы населяют мировой лист суперструны уже
в выделенной 10"=размерности и именно их присутствие делает струну суперсимметричной. 

В 10"=мерном пространстве, на достаточном расстоянии от струны возникает
суперсимметричный вариант гравитации, названный супергравитацией.
Оказалось, что супергравитация возможна только при условии,
что размерности пространства-времени находятся в пределах от 2"=х до 11"=ти.
Десятимерные теории супергравитации представляют собой предел,
к которому сводится теория суперструн на больших расстояниях,
а супергравитации в пространствах меньшей размерности получаются из десятимерных. 

Таким образом, известные ранее теории поля оказались пределом теории суперструн,
а их симметрии частью симметрии струнной теории.
Однако, 11"=мерная супергравитация представляется здесь лишней,
и поэтому не вполне понятной.

Какое же взаимодействие четырехмерной физики и теории суперструн возможно в десятимерии?
Идея взаимного влияния пространств различной размерности называется теорией Калуцы-Клейна.
Рассмотрим самый простой случай "--- приведение пятимерного мира к четырехмерному.
Для этого в пятимерии нужно рассматривать не «плоское» пространство,
а пространство, представленное в виде «цилиндра»,
т. е. считать одно из измерений свернутым в кольцо.
Скрученный в тонкую полоску лист бумаги больше похож на линию, чем на плоскость,
а линия "--- одномерное пространство.
Но все же он остается именно трубкой.
Но представим, что по этому листу бумаги движутся какие-то частицы.
Пока лист не скручен или радиус трубки не слишком мал,
эти частицы движутся во всех направлениях.
По мере того, как радиус цилиндра уменьшается,
частицы движутся вокруг трубки все быстрее и быстрее,
а их движение вдоль трубки остается без изменения и происходит с той же скоростью,
что и на плоском листе.
Если диаметр трубки приближается к размеру самой частицы, время,
за которое частица проходит полный круг настолько мало, что мы не можем его фиксировать,
нам кажется, что она движется только вдоль «плоского» направления, вдоль трубки.
Таким образом, двумерное пространство свелось к одномерному.
В действительности движение по измерениям, закрученным в кольцо,
не удаётся заметить, так как действует принцип неопределённости.
Чем меньше размеры окружности, тем больше энергии нужно затратить, чтобы частица двигалась по ней.
Поэтому, как только измерения сворачиваются в маленькие окружности, не хватает энергии,
чтобы заставить частицу двигаться по ней, таким образом, это измерение как бы исчезает. 

Мы знаем, что частицы в микромире "--- это кванты соответствующих полей,
и последовательное описание их взаимодействий осуществляется исходя из этого утверждения.
Поля могут иметь сотни различных компонент и, как правило, их тем больше, чем выше размерность пространства-времени.
Компоненты "--- это как бы отдельные поля,
но они все собраны в единую структуру и не обладают без неё абсолютной самостоятельностью.
Например, электромагнитное поле в 4"=мерном пространстве имеет четыре компоненты.
Две из них ненаблюдаемы, а другие две соответствуют двум направлениям поляризации фотона.
Если представить, что поле существует в пространстве,
одно или несколько измерений которого свернуты в маленькие окружности (или просто свёрнуты),
то есть в эффективном пространстве меньшей размерности, это поле должно будет преобразовать себя так,
чтобы число компонент уменьшилось до количества, ожидаемого от него в новом пространстве меньшей размерности.
Лишние компоненты поля при этом оказываются полностью независимыми, самостоятельными и выступают как новые поля.

Суть теории Калуцы-Клейна состоит в том, что некоторые наборы вроде бы никак
не связанных полей в четырёхмерном пространстве могут оказаться осколками
единого поля в пространстве более высокой размерности.
У существующих в 10 и 11"=мерных пространствах полей достаточно компонентов,
чтобы упаковать в них все поля, имеющиеся в четырехмерии.
Но как объяснить, почему десятимерие распалось именно на $4 + 6$ измерения,
а не, например, $3 + 7$ или $5 + 5$?

На сегодняшний день неизвестно,
как осуществляется выбор между разными вариантами скрутки и разбивки.
Однако возможности такого выбора встроены в теорию суперструн,
поскольку суперструны порождают гравитацию, которая и определяет геометрию пространства"=времени.
Можно определить, может ли то или иное шестимерное пространство быть отобранным суперструной,
чтобы из десятимерия получился наблюдаемый четырехмерный мир.
Определяющим критерием для этого служит суперсимметрия "--- не во всяком пространстве может существовать суперструна,
структура шестимерия должна быть согласована со свойствами наблюдаемого мира.
Дело в том, что при скручивании лишних измерений в очень маленькие пространства,
свойства теории в остающихся измерениях отражают некоторые геометрические характеристики этих пространств.

От наблюдаемых свойств элементарных частиц (при доступных малых энергиях в ускорителях) переходят к теории суперструн,
экстраполируя эти свойства на очень высокие энергии (не доступные пока, но существенные для струнного описания).
В рамках струнной формулировки теории ученые пытаются понять, каковы механизмы,
«переводящие» струнные сущности (иногда непосредственно не наблюдаемые, как и свойства полей, находящихся на мировом листе струны) в термины геометрии скрученных измерений,
а затем на язык четырехмерия и существующих в нем элементарных частиц.

Физические процессы описаны уравнениями, как правило с некоторыми начальными условиями.
Т. е. теоретически мы можем рассчитать поведение какой"=либо системы на длительное время,
но практически это можно сделать лишь в некотором приближении.
Для наиболее точного вычисления была сознана теория возмущений,
т. е. сначала поведение системы рассчитывается в приближении,а затем вносятся коррективы.
Однако существуют ситуации, в которых теория возмущений неприменима,
например, если необходимо рассчитать движение в системе тройной звезды,
массы звезд в которой примерно одинаковые.
Такую ситуацию называют «сильная связь» и подобные задачи решаются только с абсолютной точностью,
если их решение вообще может быть проведено. 

Проблема сильной связи есть и в теории суперструн. Прежде чем приступить к ее рассмотрению, необходимо обратить внимание на один очень важный момент: струнам доступно то, что недоступно частицам.
При наличии хотя бы одного скрученного измерения они могут «наматываться» на него, делая один или несколько витков.
С точки зрения наблюдателя это выглядит как появление некоторых новых частиц.
При определённых соотношения между радиусом свернутого измерения и количеством оборотов струны такие частицы становятся легкими,
и их можно сравнивать с теми безмассовыми частицами, появление которых ожидалось с самого начала,
как соответствующих низшим гармоникам колебаний струны.

В итоге получается, что при слабом взаимодействии между струнами, в рамках стандартной теории возмущений струна порождает определенные частицы, реализующие некоторые виды симметрии, в частности суперсимметрию. В другом диапазоне интенсивности взаимодействия, вне рамок теории возмущений (в области сильной связи) струна может порождать другие частицы.

Рассмотрим подробнее 5 существующих на сегодняшний день теорий суперструн.

Большинство удачных теорий физики элементарных частиц основываются на калибровочной симметрии. В таких теориях различные поля могут переходить одно в другое. Эти переходы полностью определяются калибровочной группой теории. Если можно провести некое калибровочное преобразование в точке пространства и при этом теория не изменится, то говорят, что теория имеет локальную калибровочную симметрию.

У струн могут быть совершенно произвольные условия на границе. Например, замкнутая струна имеет периодичные граничные условия - струна "переходит сама в себя". У открытых же струн могут быть два типа граничных условий - условия Неймана и условия Дирихле. В первом случае конец струны может свободно двигаться, правда, не унося при этом импульса. Во втором  случае, конец струны может двигаться только по некоторому многообразию. Это многообразие и называется D-браной или Dp-браной (при использовании второго обозначения 'p' - целое число, характеризующее число пространственных измерений многообразия).

D-браны могут иметь число пространственных измерений от -1 до числа пространственных измерений заданного пространства-времени. Например, в теории суперструн 10 измерений - 9 пространственных и одно временное. Таким образом, для суперструн может существовать D9-брана, но возникновение D10-браны невозможно. Отметим, что в этом случае концы струн фиксированы на многообразии, покрывающем все пространство, поэтому они могут двигаться везде, так что это сводится к наложению условия Неймана. В случае p=-1 все пространственные и временные координаты фиксированы, и такая конфигурация называется инстантоном или D-инстантоном. Если p=0, то все пространственные координаты фиксированы, и конец струны может существовать лишь в одной единственной точке в пространстве, так что D0-браны зачастую называют D-частицами. Совершенно аналогично D1-браны называют D-струнами. Кстати, само слово 'брана' произошло от слова 'мембрана', которым называют 2-мерные браны, или 2-браны. В действительности D-браны динамичны, они могут флуктуировать и двигаться. Например, они взаимодействуют гравитационно.

Используя минимально-связанную теорию возмущений, можно выделить пять различных согласованных суперструнных теорий, известных как Type I SO(32), Type IIA, Type IIB, SO(32) Гетеротическая (Heterotic) и E8 x E8 Гетеротическая (Heterotic).

\begin{table*}
    \begin{tabular}{|c|c|c|c|c|c|}
                                 & Type IIB         & Type IIA            & E8 x E8 Гетеротическая & SO(32) Гетеротическая & Type I               \\
        \hline
        Тип струны               & Замкнутые        & Замкнутые           & Замкнутые              & Замкнутые             & Открытые и замкнутые \\
        \hline
        10d Суперсимметрия       &$N=2$ (киральная) & $N=2$ (некиральная) & $N = 1$                & $N = 1$               & $N = 1$              \\
        \hline
        10d Калибровочные группы & нет              & нет                 & E8 x E8                & SQ(32)                & SQ(32)               \\
        \hline
        D"=браны                 & -1, 1, 3, 5, 7   & 0, 2, 4, 6, 8       & нет                    & нет                   & 1, 5, 9              \\ 
        \hline
    \end{tabular}
\end{table*}

\begin{enumerate}
    \item Type I SO(32):
    
    Эта теория касается открытых суперструн. В ней есть только одна (N=1) суперсимметрия в десятимерии. Открытые струны могут переносить на своих концах калибровочные степени свободы, а для того, чтобы избежать аномалий, калибровочная группа должна быть SO(32) (SO(N) - Группа N x N ортогональных матриц с определителем, равным единице. Ортогональность означает, что транспонированная матрица равна обратной). Кроме того, в ней рассмтриваются D-браны с 1,5 и 9 пространственными измерениями. 

    \item Type IIA:

    Это теория замкнутых суперструн с двумя (N=2) суперсимметриями в десятимерии. Два гравитино (суперпартнера гравитона) движутся в противоположных направлениях по мировому листу замкнутой струны и имеют противоположные киральности по отношению к 10-мерной группе Лоренца, так что это некиральная теория. Также в ней не рассматривается калибровочной группы, зато есть рассматриваются D-браны с 0,2,4,6 и 8 пространственными измерениями.

    \item Type IIB:
    
    Это тоже теория замкнутых суперструн с N=2 суперсимметрией. Однако в этом случае гравитино имеют одинаковую киральность по отношению к 10-мерной группе Лоренца, так что это киральная теория (Хиральность - свойство объекта не совпадать, не совмещаться со своим зеркальным отображением (в плоском зеркале) ни при каких перемещениях и вращениях). Снова нет калибровочной группы, но есть D-браны с -1, 1, 3, 5, и 7 пространственными измерениями. 

    \item SO(32) Гетеротическая (Heterotic):
    
    А это струнная теория с суперсимметричными полями на мировом листе, двигающимися в одном направлении, и несуперсимметричными, двигающимися в противоположных. В результате получаем N=1 суперсимметрию в десятимерии. Несуперсимметричные поля делают вклад в спектр как безмассовые бозоны, а сам спектр не аномален только из-за SO(32) калибровочной симметрии. 

    \item E8 x E8 Гетеротическая (Heterotic):
    
    s Совершенно идентична SO(32) за тем исключением, что в ней вместо группы SO(32) используется группа E8xE8, что тоже устраняет аномалии в спектре.
\end{enumerate}

Стоит отметить, что E8 x E8 Гетеротические струны исторически рассматривались как самая перспективная теория для описания физики вне Стандартной Модели. Она в течение длительного времени считалась единственной струнной теорией, имеющей хоть какое-то отношение к реальному миру. Связано это с тем, что калибровочная группа Стандартной Модели - SU(3)xSU(2)xU(1) - хорошо соотносится с одной из групп E8. Вторая E8 не взаимодействует с материей кроме как через гравитацию, что может объяснить проблему темной материи в астрофизике. Из-за того, что мы все еще не полностью понимаем струнную теорию, вопросы типа «как происходило нарушение суперсимметрии» или «почему в Стандартной Модели именно три поколения частиц», остаются без ответа. Большинство подобных вопросов имеют отношение к компактификации, которая также называется теорией Калуцы-Клейна. Пока же ясно то, что струнная теория содержит все элементы, чтобы быть теорией объединенных взаимодействий, и можно сказать, что это пока единственная настолько завершенная теория подобного толка. Однако мы не знаем, каким же образом все эти элементы описывают наблюдаемые явления. 

Кроме того, теория каждого из пяти типов суперструн говорит о том, что любая суперструна способна порождать наборы частиц, которые выглядят как соответствующие колебания суперструны другого типа. Это происходит в области сильной связи. Например, струна первого типа может в области сильной связи имитировать поведение струны второго типа, и наоборот. 

На основе этого был сделан вывод, что имеющиеся описания суперструн, все пять теорий, есть «подтеории», часть одной более общей теории, более глобальной, чем  теория суперструн. Причем она выглядит как теория суперструн только в области слабой связи, в области же сильной связи она может обнаружить совершенно новые возможности.