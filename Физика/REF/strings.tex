Говоря о фундаментальной теории, обычно подразумевают квантовую теорию, описываемую уравнениями квантовой механики.
Однако уравнения описывающие гравитационное поле (четвертое взаимодействие) "--- классические,а не квантовые.
Они служат приближением к истинным квантовым уравнениям и перестают работать,
если расстояние между объектами очень мало или их энергии слишком велики.
Классические гравитационные уравнения (в Общей Теории Относительности) на маленьких расстояниях
($\thicksim  2 \ast  10^{-35} \text{ м}$) перестают описывать реально протекающие процессы.
Однако с квантованием гравитации у ученых возникли проблемы,
решить которые им не удается и по сей день,
хотя такое явление как электромагнетизм легко квантуется.
Разрабатываемые теории содержали противоречия.
Гравитация описывает не свойства пространства-времени,
а непосредственно его физическую сущность. Для устранения противоречий,
ученые математики и физики сделали предположение о существовании струн, создав новую теорию.

Вместо точечных объектов "= частиц "---
эта теория оперирует протяженными объектами "--- струнами.
Струна не материальна, тем не менее,
ее можно представлять себе приближенно в виде некой натянутой нити, веревки, или, например,
скрипичной струны, находящейся в десятимерном пространстве-времени.
При этом надо помнить что струна "--- фундаментальный объект,
который ни из чего не состоит (ее нельзя разделить на несколько меньших объектов).
Струны могут быть замкнутыми или незамкнутыми (открытыми).
Колебания струны (как и колебания струн у гитары) могут происходить с разными частотами (гармониками),
начиная с некоторой низшей (основной) частоты. Фундаментальность открытия в том,
что на достаточно большом расстоянии от струны ее колебания воспринимаются как частицы,
и колеблющаяся струна с некоторой комбинацией основных гармоник (как и у реальной струны) порождает множество, целый спектр разных частиц.
На большом расстоянии от струны частицы выглядят как кванты известных полей "--- гравитационного и электромагнитного.
Отсюда возникает представление о том,
что частицы в квантовых теориях "--- не кусочки вещества,
а определенные состояния более общей сущности "= поля.
Масса частиц "--- полей возрастает по мере увеличения частоты породивших их колебаний.

Но зададимся вопросом "--- а является ли описание струны последовательно математическим?
Для избежания противоречия теория струн должна быть построена особым образом.
Итак: теория очень быстро приходит к внутреннему противоречию,
если размерность пространства"=времени не равна 26. 

Распространяясь в 26"=мерном пространстве "= времени, струна, как объект одномерный,
рисует поверхность, называемую мировым листом
(по аналогии с мировой линией, которую рисует частица в 4"=мерном пространстве"=времени).
Мировые листы замкнутых и незамкнутых струн различаются.
Двумерная поверхность мирового листа служит “ареной”, на которой может происходить какой"=либо процесс.
Например, на ней могут существовать двумерные (не наблюдаемые непосредственно) поля.
Свойства струны в значительной степени зависят от конкретных частиц, находящихся на мировом листе, образованном ей.
Пока струна существует в 26"=мерном пространстве"=времени, на ней ничего нет,
но если что"=то появится, она, возможно, сможет существовать в пространстве с меньшим количеством измерений.
Если рассматривать так называемую простую или бозонную струну, степени свободы возникающих на листе)
двумерных полей в определенном смысле играют роль недостающих пространственных размерностей и тем самым в пространствах меньшей размерности восстанавливают 26"=мерность.

Существуют и другие условия непротиворечивости струнной теории.
Низшие гармоники соответствуют частицам, не имеющим массы.
Оказалось, что самая низкая гармоника бозонной струны должна восприниматься как частица мнимой массы "--- тахион.
Эти частицы должны двигаться со скоростью,
превышающей скорость света, что не может не вызывать сомнений у ученых.
Появление тахионов в физической системе струны приводит к ее нестабильности, а точнее "---
тахионы очень быстро забирают из системы всю энергию и переносят ее в другие области пространства.
При их появлении можно говорить о нестабильности системы и неизбежном распаде на состояния, лишенные тахионов. 

Таким образом, теория самых простых (бозонных)
струн оказалась несостоятельной и возникла необходимость ее перестройки.