Эту, более общую, теорию назвали М"=теорией, от английского слова «Mystery» - тайна. 
Это именно та теория, различные фазы которой может описывать каждая из пяти теорий суперструн из десятимерия.
М"=теория может перейти в каждую из теорий суперструн, если она существует в пространстве с размерностью более десяти.

Сначала ученые предполагали разработать М"=теорию для 11"=мерного пространства.
В таком случае понятно, каким образом лишние, по сравнению с десятимерием степени свободы теории комбинируются в десятимерный мир, в котором существуют суперструны.
Например, одна теория получается, когда 11"=е измерение скручивается в очень маленькую окружность "--- что"=то вроде 10"=мерного цилиндра.
Другая теория возникает, когда М"=теория выделяет две десятимерные плоскости на некотором, очень малом, расстоянии друг от друга.
Эти плоскости, а точнее гиперплоскости, параллельны друг другу.
Тогда 10"=мерный мир воспроизводится граничными эффектами чего"=то более общего,
происходящего во всем объеме 11"=мерного пространства.

Оказалось, что при слабой связи и малой энергии, М"=теория превращается в 11"=мерную теорию супергравитации. 
Таким образом, последняя теория, до этого стоявшая особняком, включилась в общую картину мира.
Однако 11"=мерность может породить только две теории суперструн.
Остальные три не смогли произойти из первых двух и был сделан шаг к увеличению размерности.
Для вывода из одного источника всех теорий суперструн требуется 12"=мерное пространство,
где наряду с 10"=пространственными измерениями имеются два времени.
Но в то время как каждая из пяти теорий суперсимметрична,
никакой суперсимметрии в 12"=мерном пространстве нет.

Пять описанных выше суперструнных теорий сильно различаются
с точки зрения слабо"=связанной пертурбативной теории (теории возмущений, описанной выше).
Но на самом деле, как выяснилось в последние несколько лет,
они все связаны между собой различными струнными дуальностями.
(Назовем теории дуальными, если они описывают одну и ту же физику). 

Первый тип дуальности, которую следует обсудить, "--- Т"=дуальность.
Такой тип дуальности связывает теорию, компактифицированную на окружности радиуса $R$,
с теорией, компактифицированной на окружности радиуса $1/R$.
Таким образом, если в одной теории пространство свернуто в окружность малого радиуса,
то в другой оно будет свернуто в окружность большого радиуса, но обе они будут описывать одну и ту же физику.
Суперструнные теории типа IIA и типа IIB связаны через Т"=дуальность,
SO(32) и E8 x E8 гетеротические теории также связаны через нее.

Еще одна дуальность, которую мы рассмотрим "--- S"=дуальность.
Проще говоря, эта дуальность связывает предел сильной связи одной теории с пределом слабой связи другой теории.
(Отметим, что при этом слабо связанные описания обеих теорий могут очень сильно различаться.)
Например, SO(32) Гетеротическая струнная теория и теория Типа I S "--- дуальны в 10"=мерии.
Это означает, что в пределе сильной связи SO(32) Гетеротическая теория переходит в теорию Типа I в пределе слабой связи и наоборот.
Найти же свидетельства дуальности между сильным и слабым пределами можно, сравнив спектры легких состояний в каждой из картин и обнаружив, что они согласуются между собой.
Например, в струнной теории Типа I есть D"=струна, тяжелая при слабой связи и легкая при сильной.
Эта D"=струна переносит те же легкие поля, что и мировой лист SO(32) Гетеротической струны, так что когда теория Типа I очень сильно связана,
D"=струна становится очень легкой и мы видим, что ее описание становится таким же, как и через слабо связанную Гетеротическую струну.
Другой S"=дуальностью в 10"=мерии является самодуальность IIB струн: сильно связанный предел IIB струны это другая IIB теория, но слабо связанная.
В IIB теории тоже есть D-струна (правда, более суперсимметричная, нежели D-струны теории Типа I, так что и физика здесь другая),
которая становится легкой при сильной связи, но эта D-струна также является другой фундаментальной струной теории Типа IIB.