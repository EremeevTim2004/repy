\begin{enumerate}
    \item Type I SO(32):
    
    Эта теория касается открытых суперструн. В ней есть только одна ($N=1$) суперсимметрия в десятимерии.
    Открытые струны могут переносить на своих концах калибровочные степени свободы, 
    а для того, чтобы избежать аномалий, калибровочная группа должна быть SO(32) (SO(N) "--- 
    группа N x N ортогональных матриц с определителем, равным единице.
    Ортогональность означает, что транспонированная матрица равна обратной).
    Кроме того, в ней рассмтриваются D-браны с 1,5 и 9 пространственными измерениями. 

    \item Type IIA:

    Это теория замкнутых суперструн с двумя ($N=2$) суперсимметриями в десятимерии. 
    Два гравитино (суперпартнера гравитона) движутся в противоположных направлениях по мировому листу замкнутой струны и имеют противоположные киральности по отношению к 10"=мерной группе Лоренца, так что это некиральная теория.
    Также в ней не рассматривается калибровочной группы, зато есть рассматриваются D"=браны с 0,2,4,6 и 8 пространственными измерениями.

    \item Type IIB:
    
    Это тоже теория замкнутых суперструн с $N=2$ суперсимметрией.
    Однако в этом случае гравитино имеют одинаковую киральность по отношению к 10-мерной группе Лоренца,
    так что это киральная теория (Хиральность "--- свойство объекта не совпадать, не совмещаться со своим зеркальным отображением
    (в плоском зеркале) ни при каких перемещениях и вращениях).
    Снова нет калибровочной группы, но есть D"=браны с -1, 1, 3, 5, и 7 пространственными измерениями. 

    \item SO(32) Гетеротическая (Heterotic):
    
    А это струнная теория с суперсимметричными полями на мировом листе,
    двигающимися в одном направлении, и несуперсимметричными,двигающимися в противоположных.
    В результате получаем N=1 суперсимметрию в десятимерии.
    Несуперсимметричные поля делают вклад в спектр как безмассовые бозоны,
    а сам спектр не аномален только из-за SO(32) калибровочной симметрии. 

    \item E8 x E8 Гетеротическая (Heterotic):
    
    s Совершенно идентична SO(32) за тем исключением,
    что в ней вместо группы SO(32) используется группа E8xE8,
    что тоже устраняет аномалии в спектре.
\end{enumerate}