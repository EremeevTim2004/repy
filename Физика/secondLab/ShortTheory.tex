Вращательное движение твёрдого тела удобно описывать скалярной величиной, называемой моментом инерции тел.

По определению момент инерции $I$ тела относительно некоторой оси выражается формулой

\begin{equation}
    I = \sum_{i = 1}^{n} \Delta m_i r_i^2,
\end{equation}

где $r_i$ "--- расстояние от оси до элементарных масс $\Delta m_i$, из которых состоит всё тело,
причём суммирование производят по всем элементарным массам.

Для выяснения физического смысла понятия момента инерции вычислим кинетическую энергию $E_k$ тела,
вращающегося относительно $OO^{'}$ с угловой скоростью $\omega$.

Так как линейные скорости отдельных частиц разные, поскольку различны их расстояния $r_i$ от оси вращения,
то кинетическую энергию суммированием кинетических энергий отдельных его частиц:

\begin{equation}
    E_k = \sum_{i = 1}^{n} \frac{\Delta m_iv_i^2}{2}.
\end{equation}

Но линейная скорость $v_i$ частицы связана с угловой скоростью вращения тела соотношением $v_i = \omega r_i$,
и поэтому выражение (2) перепишется так:

\begin{equation}
    E_k = \frac{\omega^2}{2} \sum_{i = 1}^{n} \Delta m_ir_i^2 = \frac{I\omega^2}{2}.  
\end{equation}

Сравнивая формулы для кинетических энергий тела,
движущегося поступательного со скоростью $v$, и тела,
вращающегося с угловой скоростью $\omega$

\begin{equation*}
    E_k = \frac{mv^2}{2} 
    \text{ и }
    E_k = \frac{I\omega^2}{2},
\end{equation*}

легко видеть их полную аналогию по форме записи.
Точно также, если основное уравнение динамики в поступательном движении в поступательном движении есть

\begin{equation}
    F = m\frac{dv}{dt},
\end{equation}

то в динамике вращательного движения оно примет форму

\begin{equation}
    M = I\frac{d\omega}{dt}, 
\end{equation}

где $M$ "--- момент внешних сил, приложенных к телу;

$\frac{d\omega}{dt}$ "--- угловое ускорение тела.

Из формул (4) и (5) следует, что момент инерции тела во вращательном движении аналогичен массе тела в
поступательном движении и при данном моменте внешних сил определяет величину углового ускорения.

Из определения момента инерции по формуле (1) вытекает,
что произведение $\Delta m_ir_i^2$ можно рассматривать как моменты инерции масс $\Delta m_i$ относительно выделенной оси,
а полный момент тела "--- как сумма моментов инерции его частей. При этом линейные размеры масс должны быть значительно меньше расстояний $r_i$,
чтобы их можно было считать точечными массами. Поэтому вычисление момента инерции и соотношение (1) сводится по существу к вычислению интеграла,
взятого по всему объёму тела, т.е.

\begin{equation}
    V = \int_m r^2dm = \int_V \rho r^2 dV,
\end{equation}

где $\rho$  "--- плотность тела.

Для тел правильной геометрической формы этот интеграл вычисляется,
в то время как для тел произвольной формы его точное вычисление невозможно.

В некоторых частных случаях вычисление моментов инерции тел относительно
произвольных осей может быть упрощено использованием теоремы Штейнера"=Гюйгенса.
По этой теореме момент инерции тела массы $m$ относительно любой оси равен моменту инерции тела относительно оси,
параллельной первой и проходящей через центр тяжести тела,
сложенному с произведением его массы $m$ на квадрат расстояния центра тяжести от первой оси.

Действительно, пусть тело $m$ вращается с угловой скоростью $\omega$ вокруг оси $C$,
перемещаясь за некоторое время из положения $A$ в $A^{'}$.
При этом если центр тяжести находится в точке $O$ на расстоянии $R$ от оси $C$,
то он переместился в точку $O^{'}$ с линейной скоростью $v = \omega R$.
Движение тела из положения $A$ и $A^{'}$ расположим на два более простых:
первое со скоростью $v$ "--- поступательное из положения $A$ в $B$
(до совмещения центра тяжести тела с точкой $O^{'}$),
и второе со скоростью $\omega$ "--- вращательное из положения $B$ в $A^{'}$ относительно оси,
проходящей через точку $O^{'}$, параллельно оси $C$.
Кинетическая энергия для соответствующих перемещений запишется в виде

\begin{equation}
    E_{k1} = \frac{1}{2} m (\omega R)^2 
    \text{ и }
    E_{k2} = \frac{1}{2}I_0 \omega^2,
\end{equation}

где $I_0$ "---  момент инерции тела относительно оси, проходящей через центр тяжести тела.

Следовательно, полная кинетическая энергия тела имеет вид

\begin{equation}
    E_k = \frac{1}{2}m(\omega R)^2 + \frac{1}{2}I_0\omega^2.
\end{equation}

В то же время $E_k = \frac{1}{2}I\omega^2$, где $I$ "--- момент инерции тела относительно оси $C$.
Отсюда

\begin{equation*}
    \frac{1}{2}I\omega^2 = \frac{1}{2}m(\omega R)^2 + \frac{1}{2}I_0\omega^2,
\end{equation*}
\begin{equation}
    I = I_0 + mR^2.
\end{equation}

Для иллюстрации приведём формулы моментов инерций некоторых простых тел вращения радиуса $R$ и массы $m$ относительно их геометрической оси:

$I = mR^2$ "--- для кольца и полого цилиндра,
    
$I = \frac{1}{2}mR^2$ "--- для диска и сплошного цилиндра,

$I = \frac{2}{5}mR^2$ "--- для шара. 