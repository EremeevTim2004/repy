%\section{Доп задание}

Используя равнение периода колебаний для крутильного маятника $T = 2 \pi \sqrt{\frac{I}{D} }$,
выведем рабочую формулу для экспериментального определения момента инерции тела произвольной формы.

Величина $D$ зависит только от самого маятника и не зависит от грузов, которые к нему присоединяются. Поэтому сначала можно подвесить к маятнику тело с известным моментом инерции (например, цилиндр) и найти $D$, а потом подвесить тело произвольной формы и найти его момент инерции.

Выведем рабочую формулу:

$T_\text{ц} = 2 \pi \sqrt {\frac{I_\text{ц}}{D}}$. Из этой формулы $D = \frac{4 \pi^2 I_\text{ц}}{T_\text{ц}^2}$.

Подставим это значение $D$ в формулу для тела произвольной формы (T2):

$T_{T2} = 2 \pi \sqrt \frac{I_{T2}}{D} = 2 \pi \sqrt \frac{I_{T2} T_\text{ц}^2}{4 \pi^2 I_\text{ц}} = T_\text{ц} \sqrt \frac{I_{T2}}{I_\text{ц}}$.

Отсюда $I_{T2} = I_\text{ц} (\frac{T_{T2}}{T_\text{ц}})^2$.

Теперь вычислим значение момента инерции для тела Т2:

