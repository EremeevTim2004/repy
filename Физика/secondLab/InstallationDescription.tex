Определим момент инерции массивного цилиндрически симетричного тела $1$ (рис. 3).
С этой целью тело прикрепим к тонкой прямоугольной пластинке $2$,
подвешенной тонкой стальной проволок $3$ к вертикальной рамке $4$,
установленной на основании $5$.

Данная система, образующая крутильный маятник, за счёт упругих сил деформаии,
возникающих в проволоке при её закручивании, может колебаться в горизонтальной плоскости.

При повороте системы на малый угол $\alpha$ в стальной проволоке возникает раскручивающий момент $M$,
который в пределах малых отклонений пропорциоонален углу, поскольку можно считать,
что деформация носит упругий характер, подчиняющийся закону Гука. Таким образом, 

\begin{equation}
    M = \alpha D,
\end{equation}

где $D$ "--- раскручивающий момент соответствующий единице угла кручения.

Если теперь системе предоставить возможность колебаться,
то она будет совершать гармоническое движение по закону

\begin{equation}
    \alpha = \alpha_0 sin\frac{2\pi}{T}t,
\end{equation}

где $\alpha$ "--- текущий угол отклонения, $\alpha_0$ "--- максимальный угол отклонения (амплитуда колебаний),
$T$ "--- период колебаний.

Угловая скорость гармонического колебательного движения выражается

\begin{equation}
    \omega = \frac{d\alpha}{dt} = \frac{2\pi\alpha_0}{T}cos\frac{2\pi}{T}t,
\end{equation}

а её максимальное значение достигается, когда система проходит положение равновесия $(t = 0, \frac{T}{2}, 2\frac{T}{2}, 3\frac{T}{2}, \dots)$, т.е.

\begin{equation}
    \omega_0 = \frac{2\pi^2\alpha_0^2}{T^2}I.
\end{equation}

В соответствии с формулами (3) и (14) кинетическая энергия системы в положении равновесия запишется

\begin{equation}
    E_k = \frac{2\pi^2\alpha_0^2}{T^2}I.
\end{equation}

Здесь $I$ "--- момент энергии системы.
Эта энергия должна равняться потенциальной энергии системы при её максимальном отклонении от положения равновесия.
Потенциальную энергию подсчитаем, используя выражение (11)

\begin{equation}
    E_p = \int_{0}^{\alpha_0} Md\alpha = \int_{0}^{\alpha_0} \alpha Dd\alpha = \frac{D\alpha_0^2}{2}.
\end{equation}

Из условия $E_k = E_p$ легко определяется $T$ "--- периодколебания системы:

\begin{equation}
    T = 2\pi\sqrt{\frac{I}{D}}.
\end{equation}

Подвесим теперь к пластинке $2$ с обеих сторон на равных растояниях $r$
от оси вращения одинаковые цилиндрики с массой $m$.

Момент инерции системы изменится:

\begin{equation}
    I_1 = I + 2I_2,
\end{equation}

где $I_2$ "--- момент инерции цилиндра относительно оси вращения системы.

По теореме Штейна"=Гюйгенса (10) можно записать

\begin{equation}
    T_1 = 2\pi\sqrt{\frac{I_0 + 2(\frac{1}{2}mr_0^2 + mr^2)}{D}}.
\end{equation}

Решая систему уравнений (17), (20) оотносительно неизвестной величины $I$, найдём:

\begin{equation}
    I = 2m(r^2 + \frac{r_0^2}{2})\cdot\frac{T^2}{T_1^2 - T^2}
\end{equation}

При выполнении работы удобно иметь дело не с величинами $r_0$ и $r$, а с диаметрами цилиндриков  $d_0$ и расстояниями $L$ между ними.

Учитывая, что

\begin{equation}
    r_0 = \frac{d}{2}
    \text{ и } 
    r = \frac{L}{2},
\end{equation}

окончательно получаем рабочую формулу:

\begin{equation}
    I = \frac{m}{2}(L^3 + \frac{d_0^2}{2})\frac{T^2}{T_1^2 - T^2}.
\end{equation}