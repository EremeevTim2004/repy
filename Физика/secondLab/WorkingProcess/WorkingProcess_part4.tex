\item Определение относительной погрешности метода измерения:
    \begin{enumerate}
        \item Прикрепить к системе цилиндр вместо иследуемого телаю
        \item Провести операции, аналогичные операциям в п.1, и вычесть момент инерции системы с цилиндром $I_2$.
        \item Результаты измерений внести в таблицу
        \item Вычитая из значения момента инерции системы с цилиндром $I_\text{ц}$.
        \item Измерить штангерциркулем радиус цилиндра $R$;
        Определить на технических весах массу цилиндра $M$.
        \item Вычислить по известным $M$ и $R$ момент инерции цилиндра относительно геометрической оси $I_\text{ц}^1$ по формуле
        
        \begin{equation*}
            I_\text{ц}^{'} = \frac{MR^2}{2} 
        \end{equation*}
        
        \item Вычесть из значения момента инерции цилиндра $I_\text{ц}$
        изменённого с помощью крутильного маятника, значение момента инерции $I_\text{ц}^1$,
        полученного с более высокой точностью, чем $I_\text{ц}$
        \item Вычислить относительную погрешность измерений момента инерции методом крутильного маятника по формуле
        \begin{equation*}
            \frac{I_\text{ц}^{'} - I_\text{ц}}{I_\text{ц}^{'}} \cdot 100 \% 
        \end{equation*}
    \end{enumerate}