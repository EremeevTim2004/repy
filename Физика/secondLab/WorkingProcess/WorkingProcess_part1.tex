
    \item Измерение момента инерции ненагруженной системы $I_0$:
    \begin{enumerate}
        \item Отклонить пластинку 2 от положения равновесия на $10"=15^\circ$
        и предоставить системе возможность свободно колебаться
        \item После установления устойчивых колебаний системы в горизонтальной
        плоскости измерить время $t$ не менее 30 полных колебаний.
        \item По известным $t$ и числу колебаний $n$ вычислить период $T$ одного полного колебания.
        \item Подвесить к пластинке $2$ симметрично дополнительные цилиндрики и вновь определить
        период колебаний $T_i$системы с цилиндрами.
        \item Измерить линейкой расстояние между цилиндриками $L$;
        штангенциркулем измерить диаметры цилиндриков $d_0$;
        измерить массы цилиндриков на технических весах.
        \item Вычислить значения постоянного множителя $K$ в рабочей формуле (22),
        где $K = \frac{m}{2}(L^3 + \frac{d_0^2}{2})$.
        \item Вычислить момент инерции системы $I_0$ по формуле (22).
        \item Результаты измерений и вычислений внести в таблицу.
    \end{enumerate} 