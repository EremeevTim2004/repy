%\begin{table}
    \centering

    \begin{tabular}{|c|c|c|c|c|c|c|c|c|c|c|}
        \hline
        Номер Опыта & $t$, с & $n$ & $T$, c &  $\vec{T}$, c       & $t_1$, с & $n_1$ & $T_1$, с & $\vec{T_1}$, с        & $K \text{, г}\bullet\text{см}^2$ & $I_0\text{, г}\bullet\text{см}^2$ \\
        \hline
                  1 & 29     & 30  & 0,97   &\multirow{3}{*}{0,96}& 53,75    &  30   & 1,79     & \multirow{3}{*}{1,78} & \multirow{1}{*}{13067,38}        & \multirow{3}{*}{5360,02}          \\
        \hline
                  2 & 28,4   & 30  & 0,95   &                     & 53.45    &  30   & 1,78     &                       &                                  &                                   \\
        \hline
                  3 & 28,36  & 30  & 0,95   &                     & 53       &  30   & 1,77     &                       &                                  &                                   \\
        \hline
    \end{tabular}

    \caption{Измерения для не нагруженной системы $I_0$} \label{table-1}
\end{table}

\begin{table}
    \centering
    
    \begin{tabular}{|c|c|c|c|c|c|c|c|c|c|c|}
        \hline
        Номер Опыта & $t$, с & $n$ & $T$, c &     $\vec{T}$, c      & $t_1$, с & $n_1$ & $T_1$, с & $\vec{T_1}$, с         & $K \text{, г}\bullet\text{см}^2$ & $I_1\text{, г}\bullet\text{см}^2$ \\
        \hline
                  1 &  35,2  & 30  &  1,17  & \multirow{3}{*}{1,18} &   57     &  30   &  1,9     & \multirow{3}{*}{1,89}  &     \multirow{3}{*}{13067,38 }   & \multirow{3}{*}{8347,49}          \\
        \hline
                  2 &  35,17 & 30  &  1,17  &                       &   56,85  &  30   &  1,9     &                        &                                  &                                   \\
        \hline
                  3 &  35,6  & 30  &  1,19  &                       &   56,1   &  30   &  1,87    &                        &                                  &                                   \\
        \hline
    \end{tabular}

    \caption{Измерения для нагруженной системы $I_1$} \label{table-2}
\end{table}

\begin{table}
    \centering
    
    \begin{tabular}{|c|c|c|c|c|c|c|c|c|c|c|}
        \hline
        Номер Опыта & $t$, с & $n$ & $T$, c &     $\vec{T}$, c      & $t_1$, с & $n_1$ & $T_1$, с & $\vec{T_1}$, с         & $K \text{, г}\bullet\text{см}^2$ & $I_1\text{, г}\bullet\text{см}^2$ \\
        \hline
                  1 &  43,1  & 30  &  1,44  & \multirow{3}{*}{1,45} &   62     &  30   &  2,07     & \multirow{3}{*}{2,07}  &     \multirow{3}{*}{13067,38 }   & \multirow{3}{*}{12588,97}          \\
        \hline
                  2 &  44    & 30  &  1,47  &                       &   62     &  30   &  2,07     &                        &                                  &                                   \\
        \hline
                  3 &  43,5  & 30  &  1,45  &                       &   62,1   &  30   &  2,07     &                        &                                  &                                   \\
        \hline
    \end{tabular}

    \caption{Измерения для нагруженной системы $I_2$} \label{table-3}
\end{table}

\begin{table}[ht]
    \centering

    \begin{tabular}{|c|c|c|c|c|c|c|c|c|c|c|}
        \hline
        Номер опыта & $t$, с & $n$ & $T$, c &  $\overline{T}$, c       & $t_1$, с & $n_1$ & $T_1$, с & $\overline{T_1}$, с        & $K \text{, г}\cdot\text{см}^2$ & $I_0\text{, г}\cdot\text{см}^2$ \\
        \hline
                  1 & 29     & 30  & 0,97   &\multirow{3}{*}{0,96}& 53,75    &  30   & 1,79     & \multirow{3}{*}{1,78} & \multirow{3}{*}{13067,38}        & \multirow{3}{*}{5360,02}          \\
        \cline{1-4} \cline{6-8}
                  2 & 28,4   & 30  & 0,95   &                     & 53,45    &  30   & 1,78     &                       &                                  &                                   \\
        \cline{1-4} \cline{6-8}
                  3 & 28,36  & 30  & 0,95   &                     & 53       &  30   & 1,77     &                       &                                  &                                   \\
        \hline
    \end{tabular}

    \caption{Измерения для ненагруженной системы $I_0$} \label{table-1}
\end{table}



\begin{table}[ht]
    \centering
    
    \begin{tabular}{|c|c|c|c|c|c|c|c|c|c|c|}
        \hline
        Номер опыта & $t$, с & $n$ & $T$, c &     $\overline{T}$, c      & $t_1$, с & $n_1$ & $T_1$, с & $\overline{T_1}$, с         & $K \text{, г}\cdot\text{см}^2$ & $I_1\text{, г}\cdot\text{см}^2$ \\
        \hline
                  1 &  35,2  & 30  &  1,17  & \multirow{3}{*}{1,18} &   57     &  30   &  1,9     & \multirow{3}{*}{1,89}  &     \multirow{3}{*}{13067,38 }   & \multirow{3}{*}{8347,49}          \\
        \cline{1-4} \cline{6-8}
                  2 &  35,17 & 30  &  1,17  &                       &   56,85  &  30   &  1,9     &                        &                                  &                                   \\
        \cline{1-4} \cline{6-8}
                  3 &  35,6  & 30  &  1,19  &                       &   56,1   &  30   &  1,87    &                        &                                  &                                   \\
        \hline
    \end{tabular}

    \caption{Измерения для нагруженной системы $I_1$} \label{table-2}
\end{table}

\begin{table}[ht]
    \centering
    
    \begin{tabular}{|c|c|c|c|c|c|c|c|c|c|c|}
        \hline
        Номер опыта & $t$, с & $n$ & $T$, c &     $\overline{T}$, c      & $t_1$, с & $n_1$ & $T_1$, с & $\overline{T_1}$, с         & $K \text{, г}\cdot\text{см}^2$ & $I_2\text{, г}\cdot\text{см}^2$ \\
        \hline
                  1 &  43,1  & 30  &  1,44  & \multirow{3}{*}{1,45} &   62     &  30   &  2,07     & \multirow{3}{*}{2,07}  &     \multirow{3}{*}{13067,38 }   & \multirow{3}{*}{12588,97}          \\
        \cline{1-4} \cline{6-8}
                  2 &  44    & 30  &  1,47  &                       &   62     &  30   &  2,07     &                        &                                  &                                   \\
        \cline{1-4} \cline{6-8}
                  3 &  43,5  & 30  &  1,45  &                       &   62,1   &  30   &  2,07     &                        &                                  &                                   \\
        \hline
    \end{tabular}

    \caption{Измерения для нагруженной системы $I_2$} \label{table-3}
\end{table}

%$m = 53,5 \text{г}, L = 22 \text{см}, d_0 = 3 \text{см}$

$K = m / 2 \ast (L^2 + d_0^2 / 2) = 
53,5 / 2 \ast (22^2 + 3^2 / 2) = 1306\,38 \text{г} \ast \text{см}^2$

$l_0 = K \ast \vec{T}^2 / (\vec{T_1}^2 - \vec{T_2}^2) =
13067,38 \ast 0,96^2 / (1,78^2 - 0,96^2) = 5360,02 \text{г} \ast \text{см}^2$

$l_1 = K \ast \vec{T}^2 / (\vec{T_1}^2) - \vec{T_2}^2 =
13067.38 \ast 1,18^2 / (1,89^2 - 1,18 ^2) =8347,49 \text{г} \ast \text{см}^2$

$l_T = l_1 - l_0 = 8347,47 - 5360,02 = 2987,47 \text{г} \ast \text{см}^2$

$l_2 = K \ast \vec{T}^2 / (\vec{T_1}^2 - \vec{T_2}^2 =
13067,38 \ast 1,45^2 / (2,07^2 - 1,45 ^2) = 12588,97 \text{г} \ast \text{см}^2)$

$l_{\text{ц}} = l_2 - l_0 = 12588,97 - 5360,02 = 7228,95 \text{г} \ast \text{см}^2$

% подсчёт погрешностей с формулой и подстановкой

$m = 53,5 \text{ г}, L = 22 \text{ см}, d_0 = 3 \text{ см}$

$K = m / 2 \cdot (L^2 + d_0^2 / 2) = 
53,5 / 2 \cdot (22^2 + 3^2 / 2) = 13067,38 \text{ г} \cdot \text{см}^2$

$I = K \cdot \overline{T}^2 / (\overline{T_1}^2 - \overline{T}^2)$

$I_0 = 13067,38 \cdot 0,96^2 / (1,78^2 - 0,96^2) = 5360,02 \text{ г} \cdot \text{см}^2$

$I_1 = 13067,38 \cdot 1,18^2 / (1,89^2 - 1,18 ^2) = 8347,49 \text{ г} \cdot \text{см}^2$

$I_2 = 13067,38 \cdot 1,45^2 / (2,07^2 - 1,45 ^2) = 12588,97 \text{ г} \cdot \text{см}^2$

$I_T = I_1 - I_0 = 8347,49 - 5360,02 = 2987,47 \text{ г} \cdot \text{см}^2$

$I_{\text{ц}} = I_2 - I_0 = 12588,97 - 5360,02 = 7228,95 \text{ г} \cdot \text{см}^2$

% подсчёт погрешностей с формулой и подстановкой

Определение относительной погрешности метода измерения:

Масса цилиндра $M$ = 1301 г, радиус цилиндра $R$ = 3,5 см.

$I_\text{ц}^{'} = MR^2 / 2 = 1301 \cdot 3,5^2 / 2 = 7968,63 \text{ г} \cdot \text{см}^2$

Относительная погрешность метода измерений $\delta I = \frac{|I_\text{ц}^{'} - I_\text{ц}|}{I_\text{ц}^{'}} \cdot 100 \% =
\frac{|7968,63 - 7228,95|}{7968,63} \cdot 100 \% = 9,28 \%. $

Найдём абсолютную погрешность измерения $I_T$ методом крутильного маятника:

$\Delta I_\text{м} = \overline{I_T} \cdot \delta I = 2987,47 \cdot 0,0928 = 277,24 \text{ г} \cdot \text{см}^2$

Окончательное значение момента инерции тела:

$I_T = 2987,47 \pm 277,24 \text{ г} \cdot \text{см}^2$